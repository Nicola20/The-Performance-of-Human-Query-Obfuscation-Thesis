\chapter{Introduction}
Playing is part of human nature and culture \cite{huizinga}. The fact that billions of people spend several hours per week playing video games \cite{gameStatistics, numberGamers} shows how captivating games can be.
Therefore, it is no wonder, that researchers became interested in the motivational potential of video games. The idea of using "game design elements in non-game contexts" \cite{gamificationDefinition} was developed and became known as gamification. The goal of gamification is to boost the motivation and concentration of people by making (boring) tasks more game-like and therefore more enjoyable. Since the first reference of the term gamification in 2002 \cite{wirkungGamificationBuch}, the field has grown rapidly and finds usage in a wide range of applications, varying from marketing and healthcare to educational purposes \cite{ actionableGamification, gamificationExamples, gamificationHealthcareExample, gamificationHealthcareExample2}.
One of the more commonly known examples is the fitness app \textit{Zombies, Run!}, a running game that motivates users by turning every run into a mission to help people in a zombie-infested, post-apocalyptic world~ \cite{zombies}. \par
Another field of application for gamification is computer science. Especially the field of machine learning is one that has become quite important. Nowadays, machine learning is used for various applications like image or speech recognition and made the development of language assistants like Siri or Alexa possible. It is also used by big internet companies like Google or Amazon to improve their search algorithms and to provide a customized user experience. But in order to develop such intelligent systems, annotated data is needed to train the algorithms. This data must be produced by humans since computers are not able to perform this task. However, annotating or producing data is often boring and monotonous.
For this reason, computer scientists fall back on gamification to make these annotation tasks more appealing. Hence, they can collect more quality data from different people quite cost-effectively through e.g. crowdsourcing \cite{moneytaryBenefits}. \par
In this thesis, we concentrate on the appliance of gamification in relation to the field of information retrieval. Information Retrieval "is concerned with representing, searching, and manipulating large collections of electronic text and other human-language data"~\cite{informationRetrieval}. Information Retrieval systems like Google or Bing are omnipresent nowadays.
These search engines process a lot of requests from various people for a whole range of different purposes on a daily basis. While helping users to satisfy their information needs, they collect data for research purposes and to improve their algorithms to provide results specifically tailored to each individuum. Although getting customized results has its benefits, the privacy incident that occurred around AOL in 2006~\cite{aol} has shown that even anonymized data of query logs can be deanonymized when made public. The resulting danger for the privacy of users was made public by two journalists of the New York Times. In their article, Barbaro and Zeller described the case of the 62-years-old widow Thelma Arnold who could be identified through sensitive search queries like \texttt{bipolar} or \texttt{60 single men}~\cite{aol}. This is quite alarming since such search queries contain private information. These kinds of findings have motivated and still motivate computer scientists to develop algorithms that provide some kind of privacy for internet searches. Today, there exist various approaches to achieve privacy of search queries, ranging from proxy-based \cite{privateWebSearch} applications to hiding or replacing the original query by a set of dummy queries \cite{arampatzis, plausiblyDeniableSearch, trackmenot1}.\par
In this thesis, we research the ability of humans to obfuscate queries and also the strategies which they develop in the process. Furthermore, we compare the effectiveness of the obfuscated queries of humans to automatically obfuscated queries. Our findings show that humans do not perform well when it comes to hiding sensitive information needs while still retrieving relevant data.
We hope that our findings will help to improve already existing algorithms or inspire new work in the field of query obfuscation.