\chapter{Conclusion and Future Work}
The goal of this thesis is to research how effective humans can retrieve relevant web pages while obfuscating sensitive information needs as well as the strategies they pursued. To collect as much data as possible and to get more people interested in participating in our research, we decided to use gamification. Therefore, we developed City of Rebellion, a web game in which players must reformulate a given sensitive query without using any of the terms it consists of. To make this task easier, the game interface includes a list of auxiliary keywords and a web page relevant to the sensitive query.\par
Through the game's logs, we were able to gather data about the effectiveness and efficiency of players, and their strategies for obfuscating queries. We find that the obfuscated queries are on average slightly longer than the sensitive queries but still in the normal range of 2-4 terms. We study the effectiveness and efficiency of obfuscated queries belonging to five strategies (Creative, Only Suggestions, Some Suggestions, No Suggestions, and New Word). These types differ in their effectiveness and efficiency. We find that the efficiency as well as the effectiveness decrease the more players deviate from suggested obfuscation terms. If we compare the effectiveness of the player's queries with queries generated with the approach of Arampatzis et al.~\cite{arampatzis}, the automatically obfuscated queries are more effective in almost all cases.
Based on our research, we conclude that people would not be able to retrieve documents easily, if they would not be provided some kind of additional information, e.g. our suggested terms. Additionally, our results suggest that the effectiveness of obfuscated search queries depends on the size of the index, with higher effectiveness on small indexes and lower effectiveness on large indexes. 
And since commercial search engines like Google have indexed many more documents than ChatNoir or our sample, the effectiveness would probably decrease even further than in our findings.
This means that people have to rely on obfuscation techniques like the approach of Arampatzis et al.~\cite{arampatzis} if they want to hide their sensitive information needs.\par
In the future, we would like to collect more query obfuscations from different population groups. For this thesis, we only gathered data of people in the context of universities, especially from the computer science departments. However, this does not reflect the people who use search engines. Therefore, we make our game accessible to a broad spectrum of players and try to advertise it on suitable conferences. 
Furthermore, we want to add more retrieval algorithms to the game to research the effectiveness of different retrieval models and the effectiveness of obfuscated queries on larger indexes. We also consider to develop a mobile version for smartphones which could potentially lead to more players. Our collected data also holds the possibility for more research aspects like the choice of terms of players. The research possibilities are not limited to the aspects we investigated in this thesis.\par
In any case, we hope that this research will help to develop better search privacy algorithms and to comprehend human query obfuscations.