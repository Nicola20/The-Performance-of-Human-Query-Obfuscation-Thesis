\chapter{Related Work}
To research the ability of humans to obfuscate queries, we developed a web game with the purpose of collecting data about possible obfuscations of sensitive search queries. First, we motivate the need for search query data protection by regarding the privacy issues of search engines and reviewing state-of-the-art techniques in the field of query obfuscation. Afterward, we illustrate why we developed a game to collect the data by explaining the effects gamification can have on the intrinsic motivation of users and their performance. And finally, we have a look at some examples for applied gamification in the context of information retrieval.\par
The internet is an enormous collection of knowledge, used by many people in their daily lives, either for work or private issues~\cite{dailyInternet}. Given the huge rise of the web~\cite{internetUsage, internetUsage2, internetWorld}, people need search engines. However, the search queries people submit to web search engines could give away private information.
For one thing, obvious information can be revealed by queries, e.g. a query \texttt{local dating} indicates that the user is single and looks for a new partner. And through profiling, data mining techniques, or classifiers even the income of users, their gender, or age can be deduced~\cite{trackmenotWeakness, classifierPrivacyAttack}. These techniques indicate that it is important that search queries stay private and can not be assigned to specific individuals. The privacy of queries becomes especially important given recent news, which report that the police have access to query logs~\cite{police}. Yet, internet companies continue to collect log files and we have seen from the example of AOL that even anonymized query logs can be deanonymized~\cite{aol}. The AOL incident is not even the only case in which the privacy of users got compromised. In 2006, Netflix published anonymized movie-rankings data from their users. With the help of very little information, researchers were able to identify some users~\cite{netflixPaper}. Another similar case happened in 1997, in which anonymized medical records could be deanonymized with the help of a publicly available voter database~\cite{privacyExamples}.\par
This shows just how much of a security problem even anonymized data can be. To tackle the problem of compromised user privacy through search engines, different approaches have been developed with the intent to hide the information need of users~\cite{arampatzis, trackmenot1, privateWebSearch, maikPaper, plausiblyDeniableSearch, knowledge, peer}. The developed algorithms can support internet users to use the Internet as usual while protecting their sensitive information needs. These algorithms are based on different ideas but follow the two basic principles of query obfuscation. They either prevent the linkability between the user's identity and a submitted query or distort user profiles. In the following, we will take a closer look at a few of them.\par
The Private Web Search \cite{privateWebSearch} plugin for Firefox is a proxy-based approach to protect the privacy of its users. It minimizes the personal data that search engines receive in every request but still returns the normal responses from queries that one would receive without the usage of this plugin. This is achieved by using an HTTP proxy to filter HTTP requests and responses before sending or receiving them through the Tor network. The filtering process removes metadata from the HTTP requests to make them indistinguishable among different users. Furthermore, the HTTP responses are filtered for JavaScript code, cookies, plugins, etc. that could reveal information about the user. Additionally, the Tor network makes it harder to distinguish users with their IP addresses and therefore to achieve query linkability.\par
Despite the elimination of personalized data, this approach has still some weaknesses. First of all, tests showed that the usage of this plugin is about 20 times slower than a normal search request~\cite{privateWebSearch} which is partly caused by the usage of Tor~\cite{torSlow}. This is a big disadvantage when it comes to usability and could potentially prevent users from installing this plugin. Second, the Tor network itself has some weaknesses that can reduce the amount of privacy of its users when exploited~\cite{torWeakness, torNew}. It is for example vulnerable to traffic analysis attacks that can deanonymize users~\cite{torInternetSurveillanceWeakness}.
Last, even though Private Web Search anonymizes its users as much as possible, it is a huge disadvantage that it does not obfuscate the search query and thus the sensitive information need of its users.\par
Other approaches like TrackMeNot~\cite{trackmenot1, trackmenot2} and the work of Arampatzis et al.~\cite{arampatzis} use cover queries to distort user profiles.
TrackMeNot is a Firefox plugin that automatically sends randomized cover queries to search engines when the browser is open. These cover queries are derived from a list of queries that functions as a seed. New cover queries are generated from an HTTP response triggered by a randomly selected query from the list. The HTTP response is parsed for suitable phrases that can serve as a new query. After the selection of a new query, the old query that started this process is deleted. This process ensures that each user has an individual list of cover queries. To simulate as well as possible the browsing behavior of its users, TrackMeNot has different strategies for scheduling the submission of the cover queries. These strategies are the usage of randomized intervals, the analysis of users' browsing behavior by their browsing history, and the utilization of so-called Query Bursts~\cite{trackmenot1}. The Query Bursts are a method that sends a series of queries to the search engine in quick succession when the user enters an actual search request. Furthermore, newer versions of TrackMeNot also prevent the identification of cover queries through technical conditions~\cite{trackmenot1}. HTTP headers, for example, can also contain user-specific data. Therefore, TrackMeNot sets the corresponding fields in the headers of the cover queries similar to the real queries. The plugin also handles active content in HTTP responses and cookies accordingly. The advantage of this idea is that the user gets the same search results that he would otherwise see without using the plugin.\par
This looks like a solid approach to achieve privacy but research has shown that TrackMeNot is not a secure system~\cite{trackmenotsogood, trackmenotWeakness}. Even standard machine-learning classifiers are able to identify user queries with an average precision of about 49\%~\cite{trackmenotWeakness}. The work of Peddinti et al.~\cite{trackmenotWeakness} also shows that the precision of identifying user queries depends on the user and ranges from 10\% up to 100\%~\cite{trackmenotWeakness}.\par
We have seen that TrackMeNot is not good at hiding the real query. This is because the plugin does not obfuscate the real query but rather submits it to the search engine. An approach that tries to achieve search privacy by actually obfuscating the real query is the work of Arampatzis et al.~\cite{arampatzis}. Instead of using simple cover queries, the actual query is replaced by a series of less private queries. Every time a user submits a search query, a bunch of new search queries derived from the original query is created. These queries are chosen by statistical means with the help of a predefined document sample. The queries are selected in a way that satisfies different privacy needs, for example, if they are suitable to obfuscate the information need while still retrieving relevant websites. After this process, the cover queries are submitted to the search engine. To present the corresponding responses to the user, the results get merged and ranked according to the original query. In addition to this algorithm, Arampatzis et al. propose to use privacy-preserving tools like Tor to prevent the linkability between the submitted queries and the user.\par
The advantage of this approach is the fact that the real query of the user is not submitted to the search engine. But the number of derived queries that have to be statistically evaluated is comparatively large which results in a higher computation effort. Furthermore, some cover queries do not obfuscate the original query very well. One example is the obfuscation of the query \texttt{gun racks} by \texttt{gun} or \texttt{gun light}~\cite{arampatzis}. This means that, without the usage of Tor or other privacy-preserving tools, third parties could draw conclusions about the actual information need of specific users. However, as we have seen before in the paragraph about the Private Web Search, Tor has its weaknesses and is slow.\par
The keyquery approach of Fr{\"o}be et al.~\cite{maikPaper} levers out an enumeration algorithm for the creation of cover queries to improve the efficiency of Arampatzis' approach by 17\% - 19\%. Still, their evaluation with increased privacy (not allowing to obfuscate gun rack with gun) showed that only 75\% of queries can be obfuscated by their algorithm. This is on par with the state-of-the-art algorithms but is far from being perfect.\par
We have seen that a broad spectrum of approaches exists to solve the challenge of privacy in conjunction with the usage of search engines. However, all these algorithms do not achieve satisfactory search privacy and still have room for improvement. That's why we want to look at the query obfuscation problem from a completely different angle.
Instead of thinking of a new algorithm to achieve privacy for users on the web, we take a look at real users. How would they obfuscate queries to protect their sensitive information need? Are they even able to do this? To collect data for this research, we used gamification and developed a little web game in which users have to obfuscate real sensitive search queries from the AOL query logs.\par
Gamification is a tool that since 2010 has gained more and more attention from various businesses and computer science sectors~\cite{wirkungGamificationBuch}. Nowadays, gamification is a market that had an estimated worth between 3 billion and 12 billion USD in 2019 and was expected to continue growing by 30\%~\cite{statistaGamificationWorth, linkedInGamificationWorth}. Even well-known companies like Audi and forums such as StackOverflow use gamification. Audi, for example, developed an e-learning game as a virtual training for their sales personal~\cite{audi} and StackOverflow uses points or "reputation" to reward active and helpful users~\cite{stackoverflow}. This is no wonder since research has shown that gamification has a high motivational potential~\cite{wirkungGamificationBuch}. Several studies indicate that the application of game design elements can have a positive effect on the intrinsic motivation \cite{intrinsicMotivation, wirkungGamificationBuch} that is "the motivation you get by inherently enjoying the task itself"~\cite{actionableGamification}. Furthermore, intrinsic motivation has a positive impact on quantitative and qualitative performance~\cite{wirkungGamificationBuch} and so does gamification~\cite{intrinsicMotivation, wirkungGamificationBuch, actionableGamification, disassembling_gamification, gamifiedSearch, gamificationMotivation}. The benefits for performance and motivation are due to the fact that the use of game elements affects the psyche by covering certain psychological needs~\cite{actionableGamification, wirkungGamificationBuch}. Apart from the boost of intrinsic motivation, the fulfillment of the psychological needs can have positive effects on the social well-being and psychological and physiological health~\cite{wirkungGamificationBuch}. To achieve this effect, there exists a wide range of game design elements.
There is a number of publications that provide an overview on several of the design elements~\cite{gameElementsListing, actionableGamification, wirkungGamificationBuch, gamificationMotivation}. Some of these publications also deal intensively with how which game design elements affect the psyche and how they can be used wisely to achieve the best motivational effect~\cite{wirkungGamificationBuch, actionableGamification, gamificationMotivation}.\par
Research has shown that the most frequently used game design elements in gamification are points, leaderboards, and badges~\cite{actionableGamification}. Depending on the sources, levels, badges, and narratives can also be included in this list~\cite{wirkungGamificationBuch, gameCrowdsourcing}.\par
Points, badges, leaderboards, and levels focus on challenge. This means that the user has to overcome problems and therefore has to develop specific skills to make progress~\cite{actionableGamification}. Points are a mean to give feedback about the user's performance. They have an enhancing effect on the performance, and the experience of competency, and some studies indicate promotion of intrinsic motivation~\cite{wirkungGamificationBuch, intrinsicMotivation}.\\
Badges and levels represent milestones and achievements of skills. These game design elements have a beneficial impact on the user's engagement and the experience of competency~\cite{wirkungGamificationBuch}. Especially levels have a positive effect on quantitative and qualitative user performance and can prolong participation~\cite{intrinsicMotivation}.\par
Leaderboards have a similar effect to levels but their effect heavily depends on their implementation. While the comparison with other users could create a sense of challenge, encouraging users to become better and outperform other users, it can also be demotivating. This negative effect could happen if someone gets stuck at the bottom of the leaderboards without a chance of rise. Therefore, developers have to take this into consideration when adding them to their games~\cite{wirkungGamificationBuch, actionableGamification, gamificationMotivation}.\\ Performance graphs also affect the perceived competence of users~\cite{wirkungGamificationBuch, gamificationMotivation}. They provide good visual feedback about the progress and achievements of each individuum. This makes people feel smarter and motivates them to improve their skills~\cite{gamificationMotivation}.\par
The other prominent game design elements narratives and avatars have a different purpose. They influence the experience of social relatedness~\cite{gamificationMotivation} and can sometimes induce task meaningfulness~\cite{wirkungGamificationBuch}.
Avatars are a mean for users to identify themselves with their virtual counterparts and users are therefore more engaged in the game. Hereby, it is important that the outlook of the avatar can be chosen. Narratives on the other hand can give meaning to an action. Furthermore, stories can act as a collective goal and support social integration \cite{wirkungGamificationBuch}. Thus, solving the tasks presented in the game becomes more important to players and improves their performance \cite{intrinsicMotivation}.\par
We have looked at examples of the effect that individual game design elements can have. However, the actual effect heavily depends on the individual player. In game theory there are different kinds of player types that respond differently to different elements, as they have different goals or motivations~\cite{wirkungGamificationBuch}. The player type Killer, for example, loves challenges and competitions in which he can outperform other players (like leaderboards). Socialisers, in contrast, prefer game elements with a social component with which they can communicate and cooperate with other players~\cite{wirkungGamificationBuch}.\par
The positive effects of some game design elements are also used by computer scientists to collect data. Especially researchers from the field of information retrieval, where a lot of data has to be collected manually, use gamification. Various games were already developed with the intent to either gather information about the search behavior of different population groups~\cite{pageFetch1, pagefetch2, fu-finder} or to get labels for websites, images, etc.~\cite{ESPlabeling, pageHunt}. In the following, we will have a further look at some examples:\par
The ESP game is a collaborative game in which two random players must work together~\cite{ESPlabeling}. These players are shown the same image for which they have to enter short descriptive strings or labels. If at some time, both of them entered the same string, they get points. The goal hereby is to collect labels for images that can be used in search engines for better retrievability. Other games to gather labeled data, in this case labels for websites, are Page Hunt, Page Race, and Page Match~\cite{pageHunt}.
In Page Hunt, players are simply shown a web page that they have to retrieve by entering search queries. If the web page is found, points are awarded. Page Race and Page Match are versions of Page Hunt but with different game dynamics. Page Race is a competitive game in which two players play against each other. Whoever retrieves the target web page first wins and gets the points. Page Match, on the other hand, is a collaborative game. Here, players must decide whether they see the same web page as their partner based on the search queries both entered. The idea of Page Hunt was also used by other researchers to gather information about the search behavior and search abilities of people. Thus, the game Page Fetch was developed, which in principle works in same way as Page Hunt but was adapted for children~\cite{pageFetch1, pagefetch2}. The objective of this version was to investigate how children perform in search tasks. Another variant of Page Hunt is Fu-Finder, a game with the goal to measure the performance of a user's querying abilities~\cite{fu-finder}.\par
The game we developed for our research is also a competitive variant of Page Hunt that possesses some additional features and game mechanics. We will have a look at the game's structure and properties in the following chapter.