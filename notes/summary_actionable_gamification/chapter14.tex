\begin{itemize}
    \item ''The White Head Core Drives are represented by the Core Drives at the Top of the Octalysis diagram:
    \begin{itemize}
        \item Core Drive 1: Epic Meaning \& Calling
        \item Core Drive 2: Development \& Accomplishment
        \item Core Drive 3: Empowerment of Creativity \& Feedback
    \end{itemize}
    The Black Hat Core Drives are represented by the Core Drives at the Bottom of the Octalysis diagram:
    \begin{itemize}
        \item Core Drive 6: Scarcity \& Impatience
        \item Core Drive 7: Unpredictability \& Curiosity
        \item Core Drive 8: Loss \& Avoidance
    \end{itemize}
\end{itemize}

\subsection{Origins of the Theory}
\begin{itemize}
    \item ''It seemed like the games that go viral but then have shorter shelf-lives utilize Core Drives that create obsession, urgency, and addictiveness. Players would become glued to the game but then towards the Endgame Phase, the joy and fun no longer persists as strongly, yet the player mechanically continues to grind through many hours ''laboring'' through them.''
    \item ''But for the games that are quite timeless [...], when players are in the Endgame phase, there seems to be a continuous sense of wellbeing and satisfaction, just like the joy one has when playing an instrument or being called to a purpose.''
\end{itemize}

\subsection{The Nature of White Hat vs Black Hat Core Drives}
\begin{itemize}
    \item ''White Hat Core Drives are motivation elements that make us feel powerful, fulfilled, and satisfied. They make us feel in control of our own lives and actions.''
    \item In contrast, Black Hat Core Drives, make us feel obsessed, anxious, and addicted. While they are very strong in motivating our behaviors, in the long run they often leave a bad taste in our mouths because we feel we've lost control of our own behaviors.''
    \item ''The advantages of White Hat Gamification are obvious and most companies who learn my framework immediately think, ''Okay, we need to do White Hat!'' They would mostly be right, except there is a critical weakness of White Hat Motivation: it does not create a sense of urgency.''
    \item ''Black Hat Gamification creates the urgency that system designers often need to accomplish their goals and change behavior. Often this cannot be accomplished through White Hat Gamification alone.''
    \item ''If a company simply implements White Hat Gamification while the user is constantly exposed to Black Hat stimuli from other sources such as email, appointments, or distractions from Facebook, they will most likely not have the opportunity to test out the experience. Of course, this user will feel terrible also, because they will continue to procrastinate instead of doing the things that are more meaningful and make them feel good. Unfortunately, because of the nature of Black Hat motivation, they will continue behaving that way nevertheless.''
\end{itemize}

\subsection{Zynga and Black Hat Gamification}
\begin{itemize}
    \item ''With a good understanding of White Hat and Black Hat game design, you can begin to analyze and predict the strengths and longevity of any motivational system. If there aren't any Black Hat techniques, it is likely there won't be any breakouts success; if there aren't any White Hat techniques, users will quickly burn out and leave for something better.''
\end{itemize}

\subsection{Black Hat with a Clear Conscious}
\begin{itemize}
    \item ''I want to clarify here, that just because something is called ''Black Hat Gamification'' doesn't mean it's necessarily bad or unethical. Some people voluntarily use Black Hat gamification to force themselves to live healthier and achieve their short term and long term goals.''
    \item ''However, whether it is ''good'' or ''bad'' depends on the intentions and final outcome of those actions. We could use Black Hat designs to motivate people towards good behaviors or we could use Black Hat designs to motivate people towards evil.''
\end{itemize}

\subsection{When to Use White Hat Gamification Design}
\begin{itemize}
    \item ''Because of their natures, there are dominant strategies to determine when and how to use either White Hat or Black Hat gamification. Since employees motivation and workplace gamification are about long-term engagement, companies should use White Hat designs to make sure employees feel good, grow with the enterprise, and are there for the long haul.''
\end{itemize}

\subsection{When to use Black Hat Gamification Design}
\begin{itemize}
    \item ''On the other hand, when people are doing sales or running eCommerce sites, they often don't care about long-term engagement and motivation (though they probably should). All they want is for the customer to come in, buy something as quickly as possible, and then leave.''
\end{itemize}

\subsection{Bad Shifts from White Hat Design to Black Hat Design}
\begin{itemize}
    \item ''When you switch from White Hat motivation to Black Hat Motivation, you need to make sure you understand the potential negative consequences.''
\end{itemize}

\subsection{Careful Transitioning between White Hat and Black Hat}
\begin{itemize}
    \item ''In general (with some exceptions), it is better to first setup a White Hat environment to make users feel powerful and comfortable, then implement Black Hat designs at the moment when you need users to take that one Desired Action for conversation. At that point, users will likely take the Desired Action, but won't feel very comfortable. This is when you transition quickly back to White Hat motivation to make them feel good about their experience.''
\end{itemize}

\subsection{No Buyer's Remorse from TOMS}
\begin{itemize}
    \item ''The initial White Hat environment is for people to take interest and have a good opinion of your system in the first place. A venture capitalist wouldn't want to invest in a startup if he didn't first consider it world-changing and a smart investment (Core Drive 1 and 2), even if there was convincing apprehension that he may lose the deal. (Oddly enough, some investors still plunge under the pressure of Scarcity and Loss, even though they have previously determined it to be a worthless idea with no future).''
    \item ''Once people feel comfortable in your system but aren't necessarily taking the strong Desired Action, such as making a purchase, you can then use the Black Hat techniques within Core Drive 6 and 8 (and sometimes Core Drive 5), to close the deal. If the user ends up buying the product, you want to reassure them that, if true, this is indeed the smartest purchase possible (Core Drive 2), that legions of others also made the same decision (Core Drive 5), and that it positively improves the world (Core Drive 1). This will likely ensure that customers don't feel buyer's remorse.''
\end{itemize}

\subsection{What about Core Drives 4 and 5?}
\begin{itemize}
    \item ''You may have noticed that I mentioned Core Drive 4 [...] and Core Drive 5 [...] a few times in this chapter (in a White Hat context), and may wonder where they fit into all of this. They are in the middle of the Octalysis model, so are they Black Hat or White Hat?''
    \item ''Generally speaking, Core Drive 4 and Core Drive 5 have the duality of being able to be either White Hat or Black Hat. Often with Core Drive 4 [...], owning things make us feel like we are in control, that things are organized, and our general well-being is improving. We feel powerful and enriched.''
    \item ''However, sometimes the stuff we own start to own us instead. You can imagine a person who buys an extremely rare vintage car, and then becomes afraid of taking it anywhere because he is afraid to damage it or rack it up miles. at the same time, he also doesn't want to leave it at home because he's afraid it might get stolen.''
    \item ''On the other hand, for Core Drive 5 [...], we obviously enjoy and have fun when hanging out with our friends, building strong friendships, and expressing appreciation for each other. Even if we are making friends to network and build our careers (which add certain Left Brain Core Drives such as Core Drive 2 [...] as well as Core Drive 4 [...], we feel pretty positive about the experience.''
    \item ''However, sometimes peer pressure can cause some of the worst moments in our lives. When we feel pressured by our environment to behave in certain ways or get into fights with our loved ones, it starts to drive us crazy in a way that few other things can.''
    \item ''At the end of the day, each of the Core Drives wields a tremendous amount of power, and a designer must think carefully about designing for ethical purposes - to make sure there is full transparency towards the Desired Behavior, matched with the users' freedom to opt in and out. If this is not carefully done, gamification design will fail the promise of making life more enjoyable and productive, and it would simply become a source of misery and bitterness - and then likely dropped altogether.'' 
\end{itemize}