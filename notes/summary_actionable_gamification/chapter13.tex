\subsection{Left Brain vs.Right Brain Core Drives}
\begin{itemize}
    \item ''The Left Brain Core Drives involve tendencies related to logic, ownership, and analytical thoughts. They are expressed in the following three Core Drives:
    \begin{itemize}
        \item Core Drive 2: Development \& Accomplishment
        \item Core Drive 4: Ownership \& Possession
        \item Core Drive 6: Scarcity \& Impatience
    \end{itemize}
    The Right Brain Core Drives are characterized by creativity, sociality, and curiosity and as illustrated by the following:
    \begin{itemize}
        \item Core Drive 3: Empowerment of Creativity \& Feedback
        \item Core Drive 5: Social Influence \& Relatedness
        \item Core Drive 7: Unpredictability \& Curiosity
    \end{itemize}
\end{itemize}

\subsection{Extrinsic vs Intrinsic Motivation}
\begin{itemize}
    \item ''Extrinsic Motivation is motivation that is derived from a goal, purpose, or reward. The task itself is not necessarily interesting or appealing, but because of the goal or reward, people become driven and motivated to complete the task.''
    \item ''You are motivated because the extrinsic reward is extremely appealing, and it creates the illusion that you will go back to hating the task - and possibly more so than before [...].''
    \item ''Intrinsic Motivation, on the other hand, is simply the motivation you get by inherently enjoying the task itself. These are things you would even pay money to do because you enjoy doing them so much.''
    \item ''Left Brain Core Drives are by nature goal-oriented, while Right Brain Core Drives are experience-oriented. Extrinsic Motivation focuses on results, while Intrinsic Motivation focuses on the process.''
\end{itemize}

\subsection{Slight Semantic Differences with the Self-Determination Theory}
\begin{itemize}
    \item ''Here is the test I usually apply to determine if something is extrinsically or intrinsically motivated: if the goal or objective were removed, would the person still be motivated to take the Desired Action or not?''
    \item ''Motivation is what drives us to do any action, and Rewards are what we obtain once we perform the Desired Action.''
    \item ''A person may receive Intrinsic Rewards after performing a certain task, such as gaining the appreciation of others or feeling a sense of accomplishment. However, since Intrinsic Motivation is derived from the activity itself without concern for the future outcome, if a person does something for any reward, including any Intrinsic Reward, it is not based on Intrinsic Motivation.''
    \item ''This is slightly tricky to comprehend, but along the lines of Michael Wu's concepts, Core Drive 2 [...] may utilize Intrinsic Rewards, but ultimately does not focus on Intrinsic Motivation. The Left Brain Core Drives are result (goal) focused, while the Right Brain Core Drives are process (journey) focused. Core Drive 2 focuses on progress and achievements, and as a result is based on Extrinsic Motivation in my framework.'' 
\end{itemize}

\subsection{motivation Traps in Gamification Campaigns}
\begin{itemize}
    \item ''there are many motivational traps which result from using too many Extrinsic Motivation techniques at the expense of Intrinsic Motivation.''
    \item When people do something for free, then get paid for a specific time and afterwards are getting paid less or not at all, destroys their motivation and people won't do the thing they did before for free.
    \item ''many studies have shown that Extrinsic Motivation, such as paying people money to perform a task, actually lowers the creative capability to perform the task.''
    \item ''When people are thinking about the money, it distracts their focus from performance.''
    \item ''This is because when we are doing something for Extrinsic Motivators, our eyes are set on the goal, and we try to use the quickest and most effortless path possible to reach it. As a consequence, we often give up our abilities to be creative, think expansively, and refine our work.''
    \item Of course, in routine and mundane tasks that don't require any creativity and hold little Intrinsic Motivation to begin with, Extrinsic Motivation does often increase performance and results because of the goal-driven focus it generates.''
\end{itemize}

\subsection{Pay to Not Play}
    \begin{itemize}
        \item ''When a person is trying to solve the problem for free, the activity resembles play. The mind searches for new, creative ways to do things. This makes the right solution easier to find because the mind is flexible and dynamic.''
        \item ''In contrast, when a person is offered a reward, the situation immediately because one devoid of play. Unless clear, simple directions are laid out for the person, performance will actually decrease because the mind is fixated on completing the assignment.''
    \end{itemize}
    
\subsection{The Advantages of Extrinsic Motivation Design}
    \begin{itemize}
        \item ''Obviously designing for Extrinsic Motivation is not all negative. Besides enhancing a person's focus on completing monotonous routine tasks, it also generates initial interest and desire for the activity.''
        \item Often, without there being extrinsic motivation during the Discovery Phase (before people first try out the experience), people do not find a compelling reason to engage with the experience in the first place.''
        \item ''it is better to attract people into an experience using Extrinsic Rewards [...], then transition their interest through Intrinsic Rewards (recognition, status, access), and finally use Intrinsic Motivation to ensure their long term engagement.''
    \end{itemize}
    
\subsection{How to Make an Experience More Intrinsic}
    \begin{itemize}
        \item ''we've noted earlier that Intrinsic Motivation is often derived from Right Brain Core Drives, which relate to Core Drive 3, 5, and 7. Therefore, the actionable way to add Intrinsic Motivation into an experience is to think about how to implement those Core Drives into the experience.
    \end{itemize}

\subsubsection{2. Add more Unpredictability into the Experience}
    \begin{itemize}
        \item ''Unpredictability matched with Core Drive 8 [...] will often make an undesirable event even more stressful, and sometimes more motivating in a Black Hat way; but unpredictability accompanying Core Drive 2 [...] or Core Drive 4 [...] increases the excitement of the experience.''
        \item ''Like anything, there's a right way to design something, and a wrong way to design something. Ideally, if you use variable rewards, you should make sure that action to obtain them is relatively short and easy, such as pulling the lever on a slot or refreshing your Facebook home feed.''
        \item ''If you must drag out the Desired Action, it would be advisable to make sure all of the variable rewards are appealing to the users, and that the user knows that up front.''
    \end{itemize}
    
\subsubsection{3. Add more Meaningful Choices and Feedback}
\begin{itemize}
    \item ''keep in mind, our brains hate it when we have no choices but we also dislike having too many choices. The latter leads to decision paralysis and ultimately makes us feel stupid.''
\end{itemize}