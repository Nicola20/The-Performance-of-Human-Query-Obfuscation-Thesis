\subsection{And, Now it's Fun}
\begin{itemize}
    \item ''Studies have shown that we are more engaged in an experience when there is the possibility of winning than when we know our odds for certain. If we know we will receive a reward, our excitement only reflects the emotional value of the reward.''
    \item ''However, when we only have a chance to gain the reward our brains are more engaged by the thrill of whether we will win or not.''
\end{itemize}

\subsection{The Core Drive in a Skinner Box}
\begin{itemize}
    \item ''Satisfying our burning curiosity is intrinsically motivating to our primitive brain.''
    \item ''unpredictable results stemming from Core Drive 7 can drive obsessive behavior.''
\end{itemize}

\subsection{Sweepstakes and Raffles}
\begin{itemize}
    \item ''being ''lucky'' in a scenario of chance can install a higher sense of mission and purpose.''
    \item ''On a larger scale, many companies that utilize social media marketing are now successfully deploying techniques such as sweepstakes to engage users with their brand and message. Often, these companies will give out a quest where those who commit the Desired Actions will have a chance to win some promotional prize.Sweepstakes can vary quite a bit. The Desired Actions can be as simple as simple as ''liking'' the company website on Facebook.''
    \item ''Some Sweepstakes are theme-based, trying in some Core Drive 4 [...] or even Core Drive 1 [...].''
    \item Examples for the effect of Curiosity and Unpredictibility: Blendtec's Will It Blend? campaign, Adobe's Real or Fake campaign, Woot!'s Bag of Crap
\end{itemize}

\subsection{Game Techniques within Unpredictability \& Curiosity}
    \subsubsection{Glowing Choice (Game Technique \#28)}
        \begin{itemize}
            \item ''the Glowing Choice game technique makes users feel smart and competent during the Onboarding Phase. Of course, the Glowing Choice game technique also leads players in the right direction by appealing to their Curiosity.''
            \item ''Most players don't enjoy reading huge manuals or watching long videos before beginning a game; players would rather have the option to jump right in and test things out. This is where the Glowing Choice comes into play.''
            \item ''The player now knows the next Desired Action and never feels directionless.''
            \item ''In contrast to the Desert Oasis game technique [...] where the designer highlights a Desired Action by clearing out everything surrounding it, the Glowing Choice technique is about applying an overlay item that shines like a bright star in the midst of a complex environment. You can apply this method with apps by placing a strong emphasis on a key feature that represents the Desired Action that users need to be guided towards. Many apps do this by having a question mark on top of the key feature, or an arrow that points directly to what they want their potential customers to focus on.''
            \item ''The key to good design is to ensure that users don't need to think about committing the Desired Actions. In fact, users should have to think hard and decide not to take the Desired Action if they don't want to.''
        \end{itemize}
        
    \subsubsection{Mystery Boxes/Random Rewards (Game Technique \#72)}
        \begin{itemize}
            \item ''Instead of giving users Fixed-Actions Rewards where the steps to obtain them are known [...] you can build unpredictability into the experience by altering the context of how the reward is given by the nature of the reward itself.''
            \item ''In games, there is ''loot'' or ''drops'', which are random rewards that appear once the player achieves a Win-State such as opening a treasure box or defeating an enemy. Often, this unpredictable process is what drives players in the Endgame Phase. I call this technique Mystery Boxes (more gameful) or Random Rewards (more technical).''
        \end{itemize}
        
    \subsubsection{Easter Eggs/Sudden Rewards (Game Technique \#30)}
        \begin{itemize}
            \item ''Different from Mystery Boxes, Easter Eggs (or Sudden Rewards) are surprises that are given out without the user acknowledging it beforehand. In other words, where Mystery Boxes are unexpected rewards based on a certain expected trigger, Easter Eggs are rewards based on unexpected triggers.''
            \item ''Our brains are drawn to the element of surprise, and because these rewards are so unexpected the added feelings of excitement and good fortune make the experience extremely exciting. Sudden rewards incentivize customers to keep coming back in the hopes that they can inadvertently feel the same bliss again.''
            \item ''Easter Eggs are effective in two ways: They get great word-of-mouth exposure because everybody loves to share something exciting and unexpected that happened to them. Upon telling their friends about the good fortune, their friends may also become excited about the experience too.''
            \item ''Easter Eggs also create speculation on what caused the trigger in the first place. If the Easter Egg seemed to be random, participants will wonder how they can replicate the experience in order to ''game'' the system. They will start to develop theories in order to ''game'' the system. They will start to develop theories about how they won, and will commit to the assumed Desired Actions over and over again to either prove or disprove these theories.''
        \end{itemize}
        
    \subsubsection{Lottery/Rolling (Game Techniques \#74)}
        \begin{itemize}
            \item ''Another type of reward context that is fueled by Core Drive 7 [...] is the Rolling Reward, or sometimes called the ''Lottery''. The key idea of rolling rewards is the rule that somebody has to win during each period. Therefore, as long as the user ''stays in the game'', the chances of you winning increase linearly.''
            \item Example: Employee of the Week
            \item ''On the larger scale where there are a great number of participants, Rolling Reward programs have low barriers to entry and the rewards are substantial (think states or national lotteries), but there's a very slim chance to win, regardless of how long you spend playing the ''game''. Yes, individuals can increase their odds of winning by performing more of the Desired Action, such as purchasing additional tickets, or collecting additional tickets, or collecting additional entries. But again, the larger the program, the more daunting the odds.''
            \item ''The reason why lotteries work so well is because our brains are incredibly bad at distinguishing small percentages. We can't conceptually understand the difference between ''one in then million'' and ''one in hundred million''. We just register both odds as ''a very small chance'' without really comprehending that you could be winning the ''one in ten million'' prize ten times before you can win the ''one in a hundred million'' prize once!''
            \item ''as long as there is some chance, people are willing to invest small amounts of money to obtain a gigantic reward.''
            \item ''Rolling rewards work on a number of levels. For starters, because they have moderately low barriers to entry, they can easily attract a large number of participants. Furthermore, if a participant actually wins, they may easily become a fan for life. This is simply because they feel that they were chosen to win, which draws power from Core Drive 1 [...].''
        \end{itemize}