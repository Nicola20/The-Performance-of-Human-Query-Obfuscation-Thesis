\begin{itemize}
    \item ''motivates us simply because we are either unable to have something immediately, or because there is great difficulty in obtaining it.''
    \item ''We have a natural tendency to want things we can't have.''
\end{itemize}

\subsection{The Lure of being Exclusively Pointless}
\begin{itemize}
    \item ''Our brains have a natural tendency to pursue things just because they are exclusive.''
    \item E.g. rare Geomons (only 3 people have found one)
\end{itemize}

\subsection{The Leftovers aren't all that's Left Over}
\begin{itemize}
    \item ''We buy things not because of their actual value, but rather based on their perceived value, which means many times our purchases aren't very rational.''
    \item ''When there is a perceived abundance, motivation starts to dwindle. The odd thing is, our perception is often influenced by relative changes instead of absolute values.''
\end{itemize}

\subsection{Persuasively Inconvenience}
\begin{itemize}
    \item ''Our brains intuitively seek things that are scarce, unavailable, or fading in availability.''
    \item ''In his book Pitch Anything, Klaff explains the concept of Prizing, and how it ties into three fundamental behaviours of our [...] brains:
    \begin{enumerate}
        \item We chase that which moves away from us
        \item We want what we cannot have
        \item We only place value on things that are difficult to obtain.''
    \end{enumerate}
    \item ''It is oddly true that as we place limitations on something, it becomes more valuable in our minds.''
\end{itemize}

\subsection{Game Techniques within Scarcity \& Impatience}
\subsubsection{Dangling (Game Technique \#44) and Anchored Juxtaposition (Game Technique \#69)}
    \begin{itemize}
        \item ''Many social and mobile games utilize game design techniques within Core Drive 6 to heavily monetize on their users.''
        \item ''For instance, when you go on Farmville, you initially may think, ''This game is somewhat fun, but I would never pay real money for a stupid game like this.'' Then Farmville deploys their Dangling techniques and regularly shows you an appealing manison that you want but can't have. The first few times, you just dismiss it, as you inherently know it wouldn't be resource-efficient to get it. But eventually being dangled there.''
        \item ''With some curiosity now compelling you, a little research shows that the game requires 20 more hours of play before you can afford to get the mansion. [...] But then, you see that you could just spend \$5.00 and get the mansion immediately. [...] Now the user is no longer paying \$5 to buy some pixels on their screen. They are spending \$5 to save their time, which becomes a phenomenal deal.''
        \item ''An important factor to consider  when using the Dangling technique is the pathway to obtaining the reward. You have to allow the user to know that it's very challenging to get the reward, but not impossible. It is perceived as impossible, then people turn on their Core Drive 8; Loss \& Avoidance modes and go into self-denial.''
    \end{itemize}
    \textbf{Anchored Juxtaposition}
    \begin{itemize}
        \item ''With this technique, you place two options side by side: one that costs money, the other requiring a great amount of effort in accomplishing the Desired Actions which will benefit the system.''
        \item ''For example, a site could give you two potions for obtaining a certain reward: a) Pay \$ 20 right now, or b) complete a ridiculous number of Desired Actions. The Desired Actions could be in the form of ''Invite your friends,'' [...].''
        \item ''In thic case you will find that many users will irrationally choose to complete the Desired Actions. You'll see users slaving away for dozens, even hundreds of hours, just so they can save the \$ 20 to reach their goal. At one point, many of them will realize that it's a lot of time and work. At that moment, the \$ 20 investment becomes more appealing and they end up purchasing it anyway. Now your users have done both: paid you money, and committed a great deal of Desired Actions.''
        \item ''It is worth remembering that rewards can be physical, emotional, or intellectual. Rewards don't have to be financial nor do they need to come in the form of badges . people hardly pay for those. In fact, based on Core Drive 3 [...] principles, the most effective rewards are often Boosters that allow the user to go back into the ecosystem and play more effectively, creating a streamlined activity loop in the process.''
        \item ''With Anchored Juxtapositions, you must have two options for the user. If you simply put a price on the reward and say, ''Pay now, or go away'', many users will go back into a Core Drive 8 denial mode and think, ''I'm never gonna pay those greedy bastards a single dollar!'' - and then leave. Conversely, if you just put on your site, ''Hey! Please do all these Desired Actions, such as invite your friends and complete your profile!'' users won't be motivated to take the actions because they clearly recognize it as being beneficial for the system, but not for themselves.''
    \end{itemize}
    
\subsubsection{Magnetic Caps (Game Technique \#68)}
    \begin{itemize}
        \item ''Magnetic Caps are limitations on how many times a user cam commit a certain Desired Actions, which then simulates more motivation to commit them.''
        \item One ''should rarely create a feeling of abundance. The feeling of abundance does not motivate our brains. Scarcity, on the other hand, is incredibly motivating towards our actions. Even if the user committed the ultimate Desired Action by paying a lot of money, a persuasive system designer should only give people a temporary sense of abundance. After a few weeks or months, the feeling of scarcity should crawl back again with new targets for the user to obtain - currencies and needing to purchase their next batch.''
        \item ''A great system designer should always control the flow of scarcity, and make sure everyone in the system is still striving for a goal that is difficult, but not impossible, to attain. Failure to do so would cause a gratifying system to implode with users abandoning it for better grounds.''
        \item ''This plugs nicely into Mihaly Csikszentmihalyi's Flow Theory, where the difficulty of the challenge must increase along with the skill set of the user. Too much challenge leads to anxiety. Too little challenge leads to boredom.''
        \item ''There have been many [...] studies showing that by simply placing a limit on something, people's interest in it will increase. If you introduce a feature that can be used as much as people want, often few will actually use it. But once you place a use limit on the feature, more often than not, you will find people enthusiastically taking advantage of the opportunity.''
        \item ''This means that you should place a limit on an activity if you want to increase a certain behavior.''
        \item ''The best way to set a limit is to first find the current ''upper bound'' of the desired metric, and use that as the cap to create a perceived sense of scarcity but doesn't necessarily limit the behavior.''
    \end{itemize}
    
\subsubsection{Appointment Dynamics (Game Technique \#21)}
    \begin{itemize}
        \item ''Another way to reinforce this Core Drive is to harness the scarcity of time.''
        \item ''Appointment Dynamics utilize a formerly declared, or recurring schedule where users have to take the Desired Actions to effectively reach the Win-State.''
        \item ''One of the most common examples are Happy Hours, where by hitting the Win-State of showing up at the right time,people get to enjoy the reward of 50\% off appetizers and beer. People expect the schedule and plan accordingly.''
        \item They ''are powerful because they form a trigger built around time. Many products don't have recurring usage because they lack a trigger to remind the person to come back. According to Nir Eyal, author of Hooked, External Triggers often come in the form of reminder emails, pop-up messages, or people telling you to do something.
        \item ''On the other hand, Internal Triggers are built within your natural response system for certain experiences. For instance, when you see something beautiful, it triggers the desire to open Instagram.''
        \item ''With Appointment Dynamics, the trigger is time.''
    \end{itemize}
    
\subsubsection{Torture Breaks (Game technique \#66)}
    \begin{itemize}
        \item = ''not allowing people to do something immediately''
        \item ''Torture Breaks to drive obsessive behavior''
        \item ''A Torture Break is a sudden and often triggered pause to the Desired Actions. Whereas the Appointment Dynamic is more based on absolute times people look forward to''.
        \item ''Torture Breaks are often unexpected hard stops in the user's path toward the Desired Action. It often comes with a relative timestamp based on when the break is triggered, such as ''Return 5 hours from now.''''
        \item ''If the player was allowed to play for as long as they wanted - say three hours, they would likely become satisfied, stop playing, and not think about the game for a day ot two. Therefore, an omniscient game designer would perhaps allow them to play for two hours and fifty-nine minutes, and then trigger the Torture Break. At this point, they will be obsessively trying to figure out how to play that final one minute. Sometimes the game may even provide another option - ''pay \$ 1 to remove the Torture Break immediately!''''
        \item This kind of break (e.g. for 25 minutes) draws players to constantly think about those slow-passing 25 minute intervals, and makes it difficult to plan other activities while being occupied by the obsession.''
    \end{itemize}
    
\subsubsection{Evolved UI (Game Technique \#37)}
    \begin{itemize}
        \item ''The problem with most user interfaces is that they're too complex during the Onboarding stage, while too basic for the Endgame.''
        \item ''Based on the concept of Decision Paralysis, if you give users twenty amazing features at the beginning, they feel flustered and don't use a single one. But if you give them only two or three of those features (not just one, since our Core Drive 3 loves choice), and have them slowly unlock more, then they begin to enjoy and love the complexity.''
        \item Example WoW's complex user interface
        \item ''For the designer [...], it is important to acknowledge that withholding options can drive more behavior towards the Desired Action.''
    \end{itemize}
    
\subsection{Core Drive 6: The Bigger Picture}
\begin{itemize}
    \item ''Scarcity and Impatience is considered a Black Hat Core Drive, but if used correctly, it can be very powerful in driving motivation. Often, Core Drive 6 is a first source of generating Core Drive 3 [...] in the system. Overcoming scarcity can cause a higher sense of Core Drive 2.''
    \item ''When fused with Core Drive 7 [...], Core Drive 6 becomes a great engine to drive online consumer action. Finally, working alongside Core Drive 8 [...], Scarcity and Impatience becomes a powerful force that not only pushes for action with extremely strong urgency.''
\end{itemize}