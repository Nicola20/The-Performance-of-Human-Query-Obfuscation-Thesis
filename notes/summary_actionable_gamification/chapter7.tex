\begin{itemize}
    \item ''Can remove the so-called ''gamification-fatigue.'''' (Becoming tired of a game and eventually abandon it)
    \item ''Hopefully the system that you are designing actually has a purpose to it, and so even if it becomes boring (which is probably the current state anyway), your users still have a reason to stay on.''
    \item ''When a user can continuously tap into their creativity and derive an almost limitless number of possibilities, the game designer no longer needs to constantly create new content to make things engaging''
    \item Examples: Starcraft, Poker, Golf, Chess, Mahjong
\end{itemize}

\subsection{Tic-Tac-Draw}
\begin{itemize}
    \item ''When you design a great gamified system, you want to make sure there isn't one standard way to win. Instead, provide users with enough meaningful choices that they can utilize drastically different ways to better express their creativity, while still achieving the Win-State''
\end{itemize}

\subsection{The General's Carrot in Education}
\begin{itemize}
    \item ''When you design for Core Drive 3: Empowerment of Creativity \& Feedback, it is important to create a setup where the user is given a goal, as well as a variety of tools and methodologies to strategize towards reaching that goal. Often your users are not motivated because they don't understand the purpose of the activity, do not clearly identify the goal of the activity, and/or lack meaningful tools to create expressive strategies to reach the goal.''
\end{itemize}

\subsection{The Downfall of Draw Something}
\begin{itemize}
    \item Many users typically start to drop out when a game fails ''to provide fresh content and challenges that would give people a sense of continued improvement and novel conditions for further mastery.''
    \item E.g. when words repeat to often in Draw Something ''(essentially Pictionary, where one side draws something with a pen and the other side guesses)''
\end{itemize}

\subsection{Game Techniques within Empowerment of Creativity \& Feedback}

\subsubsection{Boosters (Game Technique \#31)}
    \begin{itemize}
        \item Boosters = '' a player obtains something to help them achieve the win-state effectively.''
        \item Example: mushroom or flower from Super Mario
        \item They are usually limited to certain conditions. e.g. limited time or loosing the booster when being injured
        \item They can be earned or bought
        \item The ''feeling of being empowered with new, but limited power-ups is exhilarating and an extremely strong motivator towards the desired action.''
    \end{itemize}
    
\subsubsection{Milestone Unlock (Game Technique \#19)}
\begin{itemize}
    \item ''One of the most successful design techniques within games is something that I call the Milestone Unlock. When people play games, they often set an internal stop time in the form of a milestone - ''Let me beat this boss and then I'm done.'''' e.g.
    \item ''What the Milestone Unlock does is open up an exciting possibility that wasn't there before that milestone was reached.''
    \item ''Once players level up (their ''stop-time-milestone''), they naturally want to see what these new skills are like. They will want to test them out a bit, then test them out on stronger enemies, enjoy how powerful they are, and then realize they are so close to the next milestone that they might as well get there before stopping. This is when people plan to stop at 11PM but end up playing till 4AM.
\end{itemize}

\subsubsection{Poison Picker/Choice Perception (Game Technique \#89)}
    \begin{itemize}
        \item ''Many studies have shown that people like something more when they are given a choice, compared to simply having one option. This holds true even if the multiple options are not as appealing compared to the single choice.''
        \item E.g. Your parents tell you that you have to play an instrument but you can chose which one you would like.
        \item ''The key to the Choice Perception is that the choice itself is not necessarily meaningful, but merely makes a person feel like they are empowered to choose between different paths and options.'' (E.g. you are still forced to play an instrument)
        \item ''When I say the choice is not meaningful, it could mean that either the user is presented with a good option and a bad option, inviting the user to naturally choose the better one [...]; or it could mean that all the options are too limiting and therefore undifferentiated from one another.''
        \item ''Jesse Schell, in his book \textit{The Art of Game Design - A Book of Lenses}, introduces two Lenses: The Lens of Freedom and the Lens of Indirect Control. Schell describes that, ''\textit{we don't always have to give the player true freedom. [...] if a clever designer can make a player feel free, when really the player has very few choices, or even no choice at all, then suddenly we have the best of both worlds - the player has the wonderful feeling of freedom, and the designer has managed to economically create an experience with an ideal interest curve and an ideal set of events.}''''
        \item ''According to Schell, this can be accomplished by 1) Adding constrains to player choices, 2) Incentivizing players to take certain choices that actually meets the players goals, 3) Create an Interface that guides the user towards the Desired Actions, 4) Adding visual designs to attract the player's sight, 5) Provide social guidance (often trough computer generated characters in the game), and 6) Music control that affects player behaviours.''
        \item ''Choice Perception influences our decisions in many other significant ways, such as wasting time and energy keeping meaningless doors or options open, even though they were formerly written off as bad options, simply to maintain a perception of having a choice.''
        \item ''Since Choice Perception suggests a lack of meaningful choices, it often is not ideal in an implementation as it does not truly bring out the creativity of the user. You could also offend users if too many options are blatantly meaningless.''
     \end{itemize}
     
\subsubsection{Plant Picker/Meaningful Choices (Game Technique \#11)}
\begin{itemize}
    \item ''Beyond choices that allow people to feel like they are empowered, there are choices that are truly meaningful and demonstrates preferences that are not obviously superior over others. I refer to these techniques as ''Plant Pickers'' because, just like deciding what to plant in a garden, it is often a preference on style and strategy, something that fuels Core Drive 3.''
    \item ''If you create a gamified environment with a hundred players, and all hundred of these players reach the Win-State in the exact same way (such as ''do action A, get points, do action B, get badges, do action C, win!''), there are no meaningful choices present.''
    \item ''If all hundred players play the game differently, then you have a great amount of meaningful choices.''
    \item That level of meaningful choices and play is the ultimate state of Core Drive 3: Empowerment of Creativity \& Feedback.''
    \item ''If thirty players play the game one way, thirty players another way, and the last fourty players yet another way, then you have some level of Meaningful Choices.''
    \item Examples: Plants vs. Zombies, Farmville (Art)
\end{itemize}

\subsection{Core Drive 3: The Bigger Picture}
\begin{itemize}
    \item It ''is a great Core Drive on many different levels. It taps into our innate desire to create, by providing us the tools and power to direct our own gameplay and giving us the ability to affect the environment around us trough our imagination.''
    \item Often the most difficult Core Drive to implement
\end{itemize}