\begin{itemize}
    \item ''games are a combination of behavioral economics, motivational psychology, neurobiology, UX/UI (User Experience/User Interface) design, technology platforms and [...] game design dynamics''
\end{itemize}

\subsection{Octalysis View of Self-Determination Theory}
\begin{itemize}
    \item ''The Self-Determination Theory is a theory on motivation to understand our natural or intrinsic tendencies to behave in effective and healthy ways. The theory demonstrates that people are not motivated purely through rewards and punishment, but are actually motivated more trough three elements: Competence, Relatedness, and Autonomy.''
    \item ''Competence is the need to feel self-efficacy and experience mastery. Autonomy is the urge to be casual agents of one's own life and control one's own choices. Relatedness is the universal want to interact, be connected to, and experience caring for others.''
    \item ''If you look at the theory from an Octalysis perspective, you will notice that Competence is in line with Core Drive 2 [...]. Following the same line of thought, Autonomy lends itself to Core Drive 3 [...], while Relatedness naturally falls within Core Drive 5 [...].''
\end{itemize}

\subsection{Richard Bartle's Four Player Types}
\begin{itemize}
    \item ''He realized that within a virtual environment there tends to be four main groups of players doing four distinct types of activities.''
    \item ''There are the Achievers who try to master everything there is to do within the game system. There are the Explorers that just want to go out and explore all the content in the world but aren't as focused on overcoming challenges. There are the Scocializers who are really in the virtual world just to interact with each other, have conversations, and build companionship. And there are the Killers - players that not only strive to reach the top, but take glory in beating down the competition in the process. Furthermore, they need to bask in their victories and be admired by all.''
    \item ''Richard Bartle himself has said that his Player Types may not be appropriate for environments outside of voluntary virtual worlds.''
    \item ''For simplicity's sake, let's take an Octalysis look at Richard Bartle's Four Player Types to understand what Core Drives motivate each player type. This will help you determine how to better design for these three player types.''
    \item ''\textit{Achievers} are driven heavily by Core Drive 2 [...] as well as Core Drive 6 [...]. They're always trying to complete their next goal, which makes them feel accomplished when they do. Of course, to some extent they also care about using their creativity to overcome challenges, as well as accumulating the results of their success (Core Drives 3 and 4).''
    \item ''\textit{Explorers} are dominantly motivated by Core Drive 7 [...], which drives them to discover novel content that they haven't seen before. There are also seeds of Core Drive 2, 3 and 6. They continuously use their creativity to find new ways to test every boundary that constrains them, and when they succeed, they are fulfilled by a sense of accomplishment.''
    \item ''\textit{Socializers} are primarily motivated by Core Drive 5 [...]. They like to mingle with others and bond. To a smaller extent, they are also driven to think up clever ways to engage others more (Core Drive 39, they enjoy new or unpredictable information or even gossip (Core Drive 7), and sometimes becomes territorial with their friends (Core Drive 4).''
    \item ''\textit{Killers} are primarily motivated by a mix of Core Drive 2 [...] and Core Drive 5 [...]. They not only need to strive for high goals, but they need others to recognize their accomplishments and acknowledge their superiority. In a smaller sense, they are also driven to come up with the best way to defeat the competition (Core Drive 3), avoid being killed or seen as weak themselves (Core Drive 8), and ultimately, to count their wins and victories (Core Drive 4).''
    \item ''Core Drive 1 [...] can be utilized by any of the Player Types: to achieve their goals of reaching a higher target, becoming more respected by their friends, exploring new areas, as well as defeating weaker players.''
    \item ''At the end of the day, these eight Core Drives motivate all of us to some extent, as we universally crave these Core Drives in different measures at different times. The Octalysis Framework allows us to understand whether certain Core Drives are stronger with certain people, so that we can be aware and design for these differences appropriately.''
\end{itemize}

\subsection{Nicole Lazzaro's 4 Keys To Fun}
\begin{itemize}
    \item ''The 4 Keys to Fun are: Hard Fun, Easy Fun, People Fun, and Serious Fun.''
    \item ''\textit{Hard Fun} is joy that is derived from overcoming a frustration and achieving the Win-State. This puts players in a state of Fiero, the feeling of trimph over adversity.''
    \item ''\textit{Easy Fun} is the fun from doing interesting activities where you don't need to try very hard and can simply enjoy the relaxing and playful experience. This is commonly seen in games children enjoy with their parents, such as board games or drawing.''
    \item ''\textit{Serious Fun} is fun that is engaging because it makes real world differences such as improving oneself, making more money or creating an impact in the environment.''
    \item ''Lastly, People Fun is fun that you have because you are interacting with other people and forming relationships.''
\end{itemize}

\subsection{Mihaly Csikszentmihalyi's Flow Theory}
\begin{itemize}
    \item ''The Flow theory illustrates that when the difficulty of a challenge is too high compared to a user's skill level, the result is a sense of anxiety which may compel the user to drop out quickly, Similarly, if the user's skill level is dramatically higher than the difficulty of the challenge, the user will feel bored and may also drop out.''
    \item ''Only when the user's skill level is balanced with the difficulty of the challenge, do they enter the state known as Flow. During Flow, users become completely focused. They ''zone in'' on their activities, loosing their sense of self, as well as loosing track of time. This is a moment of euphoria, excitement, and engagement.''
    \item ''The tricky thing here is that, more often than not, the player's skill level increases as time goes by. If the designer gives the exact same experience throughout the 4 Experience Phases (Discovery, Onboarding, Scaffolding, Endgame) the user quickly ends up being bored because they've outpaced the difficulty level.''
    \item ''In one of Csikszentmihalyi's charts, [...] he includes a state of ''relaxation'' as a result of having skills that greatly surpass the challenge. Core Drive 7 then does not result in boredom but rather appropriate relaxation.''
\end{itemize}

\subsection{Start Designing}
\begin{itemize}
    \item ''Think about what are the Quantifiable Metrics you want to improve. Who are the Users you are targeting and what Core Drives motivate them? What are the Desired Actions you need them to perform? What mediums can you communicate with them, show them Feedback Mechanics, and display Triggers? Finally, what are the Rewards and Incentives you can provide users?''
    \item ''Together, this becomes the Octalysis Strategy Dashboard that will be useful in any gamification design campaign.''
    \item ''More information on the Octalysis Strategy Dashboard can be found [...] at: \url{https://yukaichou.com/gamification-study/the-strategy-dashboard-for-gamification-design/}
\end{itemize}