\begin{itemize}
    \item ''There are generally two methods to use Octalysis:
    \begin{itemize}
        \item ''The first method is to analyze existing products to determine their strengths and weaknesses for motivation toward the Desired Actions. It allows us to identify what type of motivation is weak so we can introduce new improvements, often in the form of Game Design Techniques, to the experience. This is typically called an Octalysis Audit.''
        \item ''The second method is to create a brand new experience based on Octalysis and the 8 Core Drives. Trough a very systematic process, we can create an engaging experience that will fulfill the goals of the experience designer.''
    \end{itemize}
\end{itemize}

\subsection{Octalysis Review of Facebook}
\begin{itemize}
    \item ''The first step of utilizing Octalysis as a tool is to decipher all the motivation Core Drives that are present in an experience.''
\end{itemize}

\subsection{The Score is a Smoke Screen}
\begin{itemize}
    \item ''Octalysis Score is actually the least actionable element in the realm of Octalysis.''
    \item ''What's useful is using the Octalysis chart to determine which Core Drives are weak and need to be improved on, while understanding the nature of the dominant Core Drives in terms of being White/Black Hat and Left/Right Brain. The Octalysis Score itself is somewhat just fun and gimmicky.''
    \item ''Generally, for each Core Drive, we assume a number between 0 to 10 based on ''How strong does this Core Drive motivate towards the Desired Action''. 0 Means that the Core Drive does not exist as a motivator within the experience. At the top end of the scale, 10 usually means that it is impossible to improve the Core Drive further, and almost all individuals who become exposed to the Core Drive will take the Desired Action.''
    \item ''Ater that, I take the square of all eight numbers and add them together. This means that the highest possible Octalysis Score is 800 (Ten squared multiplied by eight), and the lowest is 0. In my own ratings, most successful games are above 350, and most non-game products are below 150. In fact, most products that aren't sensitive to Human-Focused Design fall below 50.''
    \item ''However, since the scoring is relatively subjective, different people will come up with different scores for the same product. When people ask me how do I know the exact numbers assigned to each motivation, I just say that it is based on an intuitive feel after experiencing the system. However, the exact numbers aren't really that useful as mentioned earlier, as long as you are not far off. It really doesn't do anything to your design whether you give a Core Drive a value of 2 or 3, as long as you know it is not a 7 or 8. As I said, Octalysis is a tool that allows people to design better based on their own sound judgements.''
    \item ''One insight that can be derived out of the Octalysis Score formula, is that it is generally better being extremely strong in a few Core Drives, as opposed to having a little bit of everything.''
    \item ''That is why it is so important to pick the right Core Drives that has the desired characteristics in your gamification design. If you are only going to pick a few Core Drives, you want to make sure that you are picking the most effective ones based on the specific scenario.''
\end{itemize}

\subsection{The Problem: Get people to drive under the speed limit}
\begin{itemize}
    \item ''In every gamified campaign, one must first define the problem to solve, and how to measure success.''
\end{itemize}

\subsection{Scaffolding Phase experience for Waze}
\begin{itemize}
    \item ''the Oracle Effect (Game Technique \#71)'' is a game technique ''where a prediction about the future causes the user to become fully engaged to see whether the ''prophecy'' will become true or not.''
    \item ''That type of curiosity adds an extra layer of fun''
\end{itemize}