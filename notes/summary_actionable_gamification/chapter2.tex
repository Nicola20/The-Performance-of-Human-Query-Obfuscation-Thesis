\subsection{Secondhand Sushi Making}
\begin{itemize}
    \item ''If you ask any gamer what makes a game fun, they will not tell you that it is because of the PBLs [(Points, Badges, Leaderboards]. They play it because there are elements of strategy and great ways to spend time with friends, or they want to challenge themselves to overcome difficult obstacles, depending on the context. This is the difference between extrinsic motivation, where you are engaged because of a goal or reward, and intrinsicmotivation, where the activity itself is fun and exciting, with or without a reward.''
\end{itemize}


\subsection{A Trojan Horse without Greek Soldiers}
\begin{itemize}
    \item ''Generic game mechanics and poorly construced game elements such as levels, boss fights or quests often fall into the same hole as PBLs. Simply put, applying traditional ''game elements'' ubiquitos in poular gameplay without diving deeper into user motivation contributes to shallow user experience [..].''
    \item Games aren't necessarily fun because of high quality graphics or flashy animations either. There are many unpoular, poor-selling games with state-of-the-art 3D high-resolution graphics.[..] Clearly, there are more to games than ''meets the eye''.
\end{itemize}

\subsection{The Story of the Good Designer vs. Bad Designer}
\begin{itemize}
    \item ''[..] a game might have all the ''right game elements'' but still be increadibly boring or stupid if they do not focus on theis users' motivation first.''
    \item Instead of starting with what game elements and game mechanics to use, the good game designer may begin by thinking ''Okay, how to I want my users to feel? Do I want them to feel inspired? Do I want them to feel proud? Should they be scared? Anxious? What's my goal for their intended experience?\\
    Once the designer understands how she wants her users to feel, then she begins to think, ''Okay, what kind of elements and mechanics can help me accomplish my goals of ensuring players feel this way.'' 
\end{itemize}