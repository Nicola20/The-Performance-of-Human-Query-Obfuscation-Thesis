\subsection{The Good Chasing the Bad}
    \begin{itemize}
        \item ''True to the nature of Black Hat motivation, you are strongly compelled to take the Desired Action, but do not necessarily feel comfortable with the behavior.''
    \end{itemize}
    
\subsection{Flipping other Core Drives Off}
    \begin{itemize}
        \item ''Core Drive 8 [...] complements many of the other Core Drives for an interesting reason: often it manifests as the reversal of the other Core Drives. You don't want something bigger than yourself to fall apart (Core Drive 1), hence you act; or you don't want to look like a loser in front of your friends (Core Drive 5), hence you make a purchase.''
        \item ''gaining something and preventing a loss is incredibly different from the standpoint of motivation. Studies have shown over and over that we are much more likely to change our behavior to avoid a loss than to make a gain. It forces us to act differently and plays by different mental rules. In fact, Nobel Prize winner Daniel Kahneman indicates that on average, we are twice as loss-averse compared to seeking a gain. This means that we have a tendency to only take on a risk if we believed the potential gain would be double the potential loss if the risk were realized.''
    \end{itemize}

\subsection{Ultimate Loss vs Executable Loss}
    \begin{itemize}
        \item This Core Drive can be tricky to manage. ''If done improperly, it can demoralize the user and lead to churn.''
        \item ''One important thing to keep in mind is that Loss \& Avoidance is motivational in a proportional manner. The way users respond to Loss \& Avoidance is generally proportional to how much they have already invested into the experience.''
        \item ''If users have played a game or used a product for ten hours, they will feel a more substantial loss than if they had invested only ten minutes. Starting over after losing is definitely more impactful on a player that's invested a few days into the game and is on level 37, as opposed to a player who just started and is on level 2.''
        \item ''The key strategy here is that the experience designer should dangle the threat of a large setback (the Ultimate Loss), but should only implement (if at all) small marginal setbacks (the Executable Loss) to emotionally train the user in taking the Ultimate Loss more seriously. The Executable Loss reinforces the avoidance.''
        \item ''As a general rule from my own experience, the Executable Loss should never be greater than 30\% of what the user has already invested in time and/or resources, and ideally never more than 15 \%. Generally a small loss of 2-5\% is enough to motivate users to take the activities seriously. If the users lose over 30\% of what they have originally invested, the odds of them quitting becomes extremely high.''
        \item ''Since it benefits no one if the user actually suffers heavy losses, it's best to utilize the ''ultimate loss'' as a form of expectation management, with the system creating ''grace systems'' that the users appreciate but do not abuse.''
        \item ''Expectations have everything to do with happiness and motivation. A hungry teenager in a poor country will have an extremely difficult time understanding why a perfectionist student in a developed country would be depressed for three weeks simply because she received a ''B'' in school. On the other hand , a student who expects to fail the class celebrates for a week when they obtain a B.''
        \item ''our happiness is almost exclusively determined by our expectations matched against our circumstances.''
    \end{itemize}
    
\subsection{A Caveat: Avoiding the Avoidance}
\begin{itemize}
    \item ''One caveat in using Core Drive 8 [...] is that the user must know exactly what they should do to prevent the undesirable event from happening.''
    \item ''if a loss-focused message is simply there by itself, but is not intuitively obvious what the user needs to do, it often backfires - the Core Drive becomes an Anti Core Drive and the user goes into denial mode.''
\end{itemize}

\subsection{Game Techniques in Loss \& Avoidance}
    \subsubsection{Rightful Heritage (Game Technique \#46)}
        \begin{itemize}
            \item ''This is when a system first makes a user believe something rightfully belongs to them, and then makes them feel like it will be taken away if they don't commit the Desired Action.''
            \item Example: Collecting points on a website but afterwards you have to sign in to save those gained points or to be able to use them.
        \end{itemize}
        
    \subsubsection{Evanescent Opportunities (Game Technique \#86) and Countdown Timers (Game Technique \#65)}
        \begin{itemize}
            \item ''An Evanescent Opportunity is an opportunity that will disappear if the user does not take the Desired Action immediately.''
            \item ''In the real world, every limited-time offer that forces you to decide whether to buy the product or lose the offer forever uses this Game Technique.''
            \item ''Evanescent Opportunities motivate us to act quickly for fear of losing a great deal. Matching well with this technique is the simple feedback mechanic called the Countdown Timer.''
            \item ''A Countdown Timer is a visual display that communicates the passage of time towards a tangible event. Sometimes the Countdown Timer is to introduce the start of a great opportunity, while at other times it's to signify the end the opportunity.''
            \item ''Earlier we mentioned that actually applying an Ultimate Loss to the user benefits no one, and that the Executable Loss is simply to make users take the Ultimate Loss more seriously. The smaller loss is meant to reinforce avoidance of the significant loss. However, if the user is not aware of the loss, the entire motivational factor is squandered.''
            \item ''Countdown Timers ensure that users recognize the presence of the Evanescent Opportunity better than a simple expiration date because the user constantly sees the window of opportunity narrowing, establishing a sense of urgency in the process. Intuitively for this purpose, Countdown Times should display the smallest time interval that is appropriate (more often than not - seconds), instead of showing longer intervals such as weeks or months.''
        \end{itemize}
        
    \subsubsection{Status Quo Sloth (Game Technique \#85) and the FOMO Punch (Game Technique \#84)}
        \begin{itemize}
            \item ''Sometimes Core Drive 8 [...] comes in the form of simply not wanting to change your behavior. I call this lazy tendency of behavioral inertia Status Quo Sloth.''
            \item ''As experience designers, our goal is to build Status Quo Sloth into the Endgame phases of our products by developing highly engaging activity loops that allow the user to turn Desired Actions into habits.''
            \item ''Nir Eyal, an expert in building habit-forming products. developed the Hook Model to describe a cycle of Triggers, Actions, Rewards, and Investments that eventually attract users into performing daily activities without exerting any metal effort. In fact, once an activity becomes a habit, users actually need to spend consistent mental and emotional energy before they can remove themselves from the habit permanently.''
            \item ''This Hook Model focuses on creating internal and external triggers that remind the user to take the Desired Actions on a daily basis.. After the user commits the Desired Action, a variable (and often emotional) reward is provided, finally prompting the user to input an investment, where the user will build value for themselves when they return again via the next trigger. Investments are things like adding a photo, tagging friends, customizing folders, where the user builds value into this process (aligning with Core Drive 4 [...]).''
            \item ''If done correctly, users begin to feel motivated by Status Quo Sloth, which means they may even work harder to prevent a change in behavior.''
            \item ''On the other hand, in order to counter the Status Quo Sloth that is working against you, something I call the ''FOMO-Punch'' might be implemented. FOMO stands for ''Fear of Missing Out'' and it's trick is to apply Core Drive 8 [...] itself.''
            \item ''In life, we fear losing what we have, but we also fear losing what we could have had. This fear of regret, when prompted correctly, can penetrate through the behavioral inertia of Status Quo Sloth and trigger the Desired Action.''
            \item ''As the context suggests, FOMO Punches can be very effective in the Discovery Phase of an experience when users are trying out a new experience. In contrast, the Status Quo Sloth technique plays a bigger role in the Endgame phase when the designer wants to keep the veterans in the system.''
        \end{itemize}
        
    \subsubsection{The Sunk Cost Prison (Game Technique \#50)}
        \begin{itemize}
            \item ''Perhaps the most powerful and sometimes treacherous mechanism within Core Drive 8 [...] is what I call the Sunk Cost Prison. This occurs when you invest so much time into something, that even when it's no longer enjoyable, you continue to commit the Desired Actions because you don't want to feel the loss of giving up on everything.''
            \item ''From a design standpoint, if you make sure the user is accumulating - and knows that they are accumulating - things that will be gone and wasted if they leave your system, it will be very difficult for the user to leave during the Endgame.''
            \item ''Sunk Cost Prison, though powerful, adhere to the Black Hat principles of making users feel uncomfortable. As such, they should always be accompanied by White Hat Core Drives, (such as allowing users to recognize that they are actually helping the world and they shouldn't give up the impact accumulated to that point). These technique should only be employed when the user has a quick urge to leave the system, such as being attracted to Black Hat technique used by other companies.''
            \item ''When you design your experience, you should think regularly about what makes users reluctant to let go and therefore stay in your system for longer.''
        \end{itemize}