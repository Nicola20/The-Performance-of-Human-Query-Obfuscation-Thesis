\begin{itemize}
    \item ''involves activities inspired by what other people think, do, or say.''
    \item It ''is the engine behind many themes such as mentorship, competition, envy, group quests, social treasure, and companionship.''
\end{itemize}

\subsection{The Mentor that Stole My Life}
\begin{itemize}
    \item assigning mentors to new users of the game helps them to understands the game easier and also motivates them to use your game just because they now have a role model.
    \item When a new user sees his/her mentor e.g. defeat monsters easily they are likely to think that he/she wants to be able to do this one day.
    \item That is because when we see someone else effortlessly complete something that we struggle hard against, our brains automatically develop a feeling of envy. How people deal with that envy may be different - some become inspired with ''I want to be like that one day!'', whereas others enter into denial, ''Well, I can never do that, but the whole thing is stupid anyway.''
    \item This feeling of ''I want to be like that one day'' could even motivate people even though they did not really care about the game before.
    \item ''When you design an environment where people are prone to be envious of others, you want to make sure there is a realistic path for them to follow to in achieving what they are envious about. Otherwise it will simply generate user denial and disengagement.''
\end{itemize}

\subsection{We're all Pinocchios at Heart}
\begin{itemize}
    \item ''While there are some people in the world that go out of their way to be different, unique, or weird [...], most people benchmark themselves with what everyone else is doing.''
    \item Many studies have shown that if you are shown/told (e.g. via a sign or advertisement) what the social norm is, what most people do, then you are very likely to adapt to this behavior even though you might have acted better before. E.g. a campaign that wants to warn you about pollution and wants you to take more responsibility in that matter (e.g. reducing pollution or saving energy). When this campaign contained the information that a lot of people are e.g. taken part in the pollution then an increase of pollution was found after the campaign went viral.
    \item ''What we perceive as the social norm greatly influences our decisions and behavior, often more so than personal gains or even moral standards.''
\end{itemize}

\subsection{The Average Person is Above Average}
\begin{itemize}
    \item ''There have been numerous studies on how ''social norming'' affects our behaviors. Often, when we see how other people are performing, we begin to compare ourselves to the norm and start to adjust accordingly. Our social standing among our peers turns out to be a strong motivator for us regardless of whether we think others recognize our standings or not.''
    \item ''When researchers study how people perceive themselves relative to others, the majority consider themselves to be ''above average'' at almost anything you ask them about. This [..] is statistically impossible.''
    \item ''The ''relatedness'' principle indicates that the more you can relate to a group, the more likely you will comply with its social norm.''
    \item As the research suggests, simply informing your users on how their fellow ''elite'' users are behaving is a simple way to significantly boost the Desired Actions towards that activity.''
\end{itemize}

\subsection{Corporate Competition as an Oxymoron}
In most of the cases creating competition in the workplace intentionally is not good for the company, on the contrary it could even be really bad.
\subsubsection{So how to implement competition correctly? }
Mario Herger suggests to consider the following when designing competition into the workplace.\\
\textbf{When competition works:}
\begin{itemize}
    \item In situations where players aim to achieve mastery of the task
    \item In gain-oriented scenarios and mindsets where players focus on becoming the winner.
    \item When contestants reach their individual Zone of Optimal Functioning (IZOF), which means their anxiety and arousal level reaches a heightened degree of focus.
    \item When players care about the welfare of the team
    \item When players are primed to overcome obstacles, and not what they will do after reaching the goal
    \item With situational anger to confrontations
    \item When there is an even matchup and players feel that they actually have a chance
    \item When players care about the competitors (competing against your friends instead of every stranger in the world.)
\end{itemize}
\textbf{Competition does not work:}
\begin{itemize}
    \item In learning-focused environments
    \item In prevention-oriented situations and attitudes where players focus on not being the loser
    \item When teams are too harmonious and competition becomes awkward
    \item When creativity is required
    \item When the competition is regarded as skewed and there is little chance to win
\end{itemize}

\subsection{Game Techniques within Social Influence \& Relatedness}
\subsubsection{Mentorship (Game Technique \#61)}
    \begin{itemize}
        \item This game technique was mentioned before in \textit{The Mentor that Stole My Life}.
        \item It is ''a consistently effective tool in every medium of activity that requires sustained motivation.''
        \item The mentor does not only provide ''directional guidance, but also emotional support to make sure the time-consuming process of pledging becomes more bearable. This practice has endured for over a century and shown to improve the Onboarding experience of members joining the organization.''
        \item ''The other benefit for Mentorship is that it also helps veteran players stay engaged during the endgame phase.''
        \item ''Poeple in general would love to have the opportunity to converse with experienced mentors who can not only help solve their interface problems, but also serve as great exemplars who they can aspire to become.''
    \end{itemize}
    
\subsubsection{Brag Buttons (Game Technique \#57) and Trophy Shelves (Game Technique \#64)}
    \begin{itemize}
        \item ''Bragging is when a person explicitly and vocally expresses their accomplishments and achievements, whereas a Trophy Shelf allows a person to implicitly show of what they have accomplished without really saying it.''
        \item ''intuitevly encouraging users to brag about and show off their achievements comes in handy when it comes to recruiting new players and keeping veteran players active, but the two techniques are appropriate for different scenarios.''
        \item ''A Brag Button is a Desired Action that users can take in order to broadcast what they feel accomplished about [...]. In other words, Brag Buttons are little action tools and mechanisms for users to broadcast how awesome they are.''
        \item Example: Temple Run - Here, ''whenever a game is over, there is a quick and easy way for users to tap a button and share a screenshot of their high score on Facebook, Instagram, and Twitter.''
        \item ''You want to implement the Brag Buttons at the Major Win-States when users actually feel awesome about what they have just accomplishes.''
        \item ''A Trophy Shelf on the other hand, is an obvious display that exhibits the achievements of the user. In other words, the user simply has to put up the Shelf, and further promotion, everyone who comes by will see and acknowledge these great achievements.''
        \item ''Trophy Shelf can often be seen as crowns, badges, or avatars. In many games, some avatar gear or items can only be obtained after reaching difficult or exclusive milestones[...]. This allows everyone to clearly see that this user has achieved a lot without the person annoyingly bringing it up all the time.''
        \item ''Keep in mind that there needs to be some level of Relatedness when someone brags about or shows off something. When there is a mutual understanding of the difficult work required to reach a certain level, people are more likely to brag about or show off their scores because they know others recognize how hard it was to obtain the scores.''
    \end{itemize}
    
\subsubsection{Group Quest (Game Technique \#22)}
    \begin{itemize}
        \item ''Group Quests are very effective in collaborative play as well as viral marketing because it requires group participation before any individual can achieve the Win-State.''
        \item ''To solve some problems or at least to make them easier you have to do it in a group. ''This motivate[s] many players to group together [e.g.] into clans and guilds[...]. Because of this people [are] encouraged to login regularly and not drop due to the social pressure.''  
        \item Example: World of Warcraft (WoW), Farmville, Groupon
    \end{itemize}
    
\subsubsection{Social Treasures (Game Technique \#63)}
    \begin{itemize}
        \item ''Social Treasures are gifts or rewards that can only be given to you by friends or other players.''
        \item E.g. in Farmville the ''only way to obtain these items was to have friends click the ''Give to Friend'' button and have them send to you.''
        \item ''This of course, pushed people to get their friends to join the game, so they could get more opportunities to obtain these Social Treasures.''
    \end{itemize}
    
\subsubsection{Social Prods (Game Technique \#62)}
    \begin{itemize}
        \item ''A Social Prod is an action of minimal effort to create a social interaction, often a simple click of a button.''
        \item ''The advantage of a Social Prod is that the user does not need to spend time thinking about something witty to say, or worry about sounding stupid when they just want to have a quick basic interaction.''
        \item Examples: Facebook Pokes, Google's +1s
    \end{itemize}
    
    \subsubsection{Conformity Anchor (Game Technique \#58)}
    \begin{itemize}
        \item ''Earlier we learned about the power of Social Norming. A game design technique I call Conformity Anchors implements this effect into products or experiences by displaying how close users are to the social norm through Feedback Mechanics.''
        \item Example: ''Opower has discovered, the best way to motivate households to consume less energy is to show them a chart comparing them to their neighbors.''
    \end{itemize}
    
\subsubsection{Water Coolers (Game Technique \#55)}
    \begin{itemize}
        \item ''In American corporate office culture, the water cooler is often the place where people take a small break from work and chat about a variety of non-work related topics. [..] gets employees to bond with one another.''
        \item ''One example of a Water Cooler is adding a forum to your site. Forums are very helpful for getting a community to bond and share ideas with each other. For this purpose, it also provides an environment to broadcast a social norm, while also connecting veterans to newbies for Mentorship opportunities.''
        \item ''One thing to take note of is that when you introduce a forum-like Water Cooler into your system, it could easily become plagued by constant inactivity. Generally forums are not very good at creating a community, but are good at mingling within an already-established community. if people visit a new forum and see that it's relatively empty, it reinforces a negative social proof.''
        \item ''It's important to first create a strong community that already has a lot to discuss and then introduce the Water Cooler to unleash the social energy.''
        \item
    \end{itemize}
    
\subsection{Core Drive 5: The Bigger Picture}
\begin{itemize}
    \item It ''adds more fun to Core Drive 3 as well as Core Drive 7. It makes Core Drive 1 more meaningful and Core Drive 2 feel more like an accomplishment. Also it gives people a benchmark on how they are doing with Core Drive 4, as well as creates envy when others have what they don't have (Core Drive 6).''
    \item ''However, there is such a thing as oversaturation: where many platforms abuse the mechanics for mass friend-spamming, losing the original social purpose of fun and collaboration. If your friends all feel like everything you do is for Left-Brain self-gain purposes, you will begin to lose their trust and be treated like a network marketer who is simply cashing out friends for money.''
\end{itemize}