\begin{itemize}
    \item ''When it comes to Epic Meaning \& Calling, it's not about what you want as an individual nor about what makes you feel good. Individuals participate in the system and take action not because it necessarily benefits them, but because they can see themselves as heros of a grander story. It's about playing your part for the greater good.''
    \item Core Drive 1 ''is generally best communicated during the Discovery and Onboarding Phase of a Player's Journey. You want to communicate very eraly on exactly why the user should participate in your mission and become a player.''
    \item Examples:
    \begin{itemize}
        \item Apple selling a vision
        \item Waze using a metaphor for traffic - picturing it as a monster and asking people to help them defeating it
    \end{itemize}
    \item ''The powerful thing about Epic Meaning \& Calling is that it turns otherwise passive users into powerful evangelists of your mission. They are even highly forgiving of your flaws.''
    \item important factor for Core Drive 1: ''It must feel authentic''
\end{itemize}

\subsection{Game Techniques within Epic Meaning \& Calling}
    \subsubsection{Narrative (Game Technique \#10)}
        \begin{itemize}
            \item ''One of the [...] effective ways to instill Epic Meaning \& Calling into your user base is trough an engaging Narrative. This allows you to introduce a story that gives people context for a higher meaning trough interacting with your company, product or web-site.''
            \item Example: Zamzee, a ''wearable technology''
        \end{itemize}
        
    \subsubsection{Humanity Hero (Game Technique \#27)}
    \begin{itemize}
        \item ''If you can incorporate a world mission into your offerings, you can gain even more buy-in during the Onboarding process.''
        \item Examples:
        \begin{itemize}
            \item TOM's Shoes: Sends one pair of shoes to a child in a third-world country whenever you place an order with them.
            \item FreeRice.com: Donates 10 grains of rice for every correct answer to the educational questions posted on their site.''
        \end{itemize}
    \end{itemize}
    
    \subsubsection{Elitism (Game Technique \#26}
    \begin{itemize}
        \item ''Allowing your users [...] to form a prideful group based on ethnicity, beliefs, or common interets also makes them feel like they are part of a larger cause. Elitism instills group pride, which means each member tries to secure the pride of the group by taking specific actions. The group also attempts to frustrate rivals, which can lead both groups upping their actions to beat the competition.''
        \item Examples
        \begin{itemize}
            \item Rivalry between schools/universities/sport clubs
            \item Kiva.org: Comparison between Christians and Atheists - who donates more?
        \end{itemize}
    \end{itemize}
    
    \subsubsection{Beginer's Luck (Game Technique \#23)}
    \begin{itemize}
        \item ''Begginer's Luck focuses part [...]. Calling makes people think they are uniquely destined to do something. With Begginer's Luck, people feel like they are one of the few chosen to take action - which makes them much more likely to take it.''
    \end{itemize}
    
    \subsubsection{Free Lunch (Game Technique \#24)}
    \begin{itemize}
        \item ''Giving freebies (that are normally not free) to select people in such a way that it binds them to a larger theme can make customers feel special and encourage them to take further action'.'
    \end{itemize}
    
    \subsection{Believability is Key}
    Even though Epic Meaning \& Calling is powerful [..], it can also backfire and fail in epic proportions. As you use these concepts, keep in mind, that you can really turn people off when you're appearing disingenuous in your efforts to create Epic Meaning and Calling.''
    
    \subsection{Core Drive 1: The Bigger Picture}
    \begin{itemize}
        \item ''It underlines the purpose behind the activity and strengthens all the other seven Core Drives when it is introduced correctly.''
        \item Its ''weakness lies in the difficulty of implementing believability, as well as the lack of urgency within the motivation. While people constantly aspire tobecome part of something bigger and would feel great if they actually took the actions, they will often procrastinate and delay those very actions. Thus, to create desirable behaviour, the gamification designer needs the help of the other Core Drives[...].''
        \item TEDx talk on how eight different world-changing products utilize each of the 8 Core Drives can be accessed at \url{https://www.youtube.com/watch?v=v5Qjuegtiyc}
    \end{itemize}