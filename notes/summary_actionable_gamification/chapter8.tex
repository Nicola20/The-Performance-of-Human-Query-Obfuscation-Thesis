\begin{itemize}
    \item ''It represents the motivation that is driven by our feelings of owning something, and consequently the desire to improve, protect, and obtain more of it.''
    \item ''Core Drive 4 is connected to our investment of time or resources into customizing something to our own liking.''
    \item ''Here decisions are mostly based on logic and analysis, reinforced by the desire for possession as the primary motivating factor.''
    \item Example: Farmville
\end{itemize}

\subsection{Wait, it's mine? Hold on, I do care then!}
\begin{itemize}
    \item ''Once you feel a sense of ownership over something, its status elevates and it begins to motivate your behavior differently.''
\end{itemize}

\subsection{Stamps of Sanity}
\begin{itemize}
    \item ''One of the most common manifestations of the Ownership \& Possession Core Drive is the desire to collect things, such as stamps or baseball cards.''
    \item ''they are only meaningful because they represent a piece in a larger set.''
    \item ''Some readers might feel that Core Drive 4 only entails the actions of accumulating more possessions, but it also provides emotional comfort to those who simply dwell upon those possessions.''
    \item ''It not only has the ability to engage, it has the ability to comfort and instill a sense of well-being. You see this phenomenon with baseball cards, pens [...]. People just like to display them and marvel at the collection for hours, while seemingly ''having fun'' in the process.''
\end{itemize}

\subsection{The Perfect Pet}
\begin{itemize}
    \item ''When people feel like they have ownership of something, they naturally want to care for it and protect it.''
    \item Example: Pet Rock
\end{itemize}

\subsection{The First Virtual Pet}
\begin{itemize}
    \item The ''ingrained sense of Ownership \& Possession, along with some added benefits of novelty (Core Drive 7: Unpredictability \& Curiosity), made products like Pet Rocks, Tamagotchi, and later on Facebook games like Farmville or Pet Society such big successes worldwide.
\end{itemize}

\subsection{The Endowment Effect}
\begin{itemize}
    \item ''When a person starts to own something, they immediatly place more value on that item relative to others who don't own it.''
    \item Examples:
    \begin{enumerate}
        \item A ''well-respected academic and wine lover becomes very reluctant to sell a bottle of wine from his collection for 100\$, but would also not pay more than 35\$ for a wine of similar quality.''
        \item Research were students had to go through a demanding process of obtaining tickets for a basketball game. They had to camp outside and pay attention to specific signals for a hole semester only to be ''lucky'' to retrieve a lottery ticket. Then the lottery was performed, deciding the winners of the tickets. Afterwards people who didn't get a ticket were asked how much they would pay for a ticket. On average they were willing to pay 170\$. On the other hand, people who won a ticket were asked for how much they would sell it. They were demanding 2,400 \$ on average, even though they had gone through the same troubles as the other to gain a lottery ticket.
    \end{enumerate}
    \item ''The Endowment Effect is also realized when we simply imagine ourselves owning something.'' It was found out that ''within auction sites, the longer people remained the top bidder (having imagined themselves as the official owner for longer), the more aggressively they would bid when someone offered a counter bid. The envisioned ownership actually motivates people to fight for their divine rights to that item they don't yet own.''
\end{itemize}

\subsection{For Sale Not For Use}
\begin{itemize}
    \item ''One caveat to the Endowment Effect is, if the person owns something as a token ''for exchange'', they will not feel attached to the item in a biased matter.''
    \item Example: ''If a merchant owns hundreds of shoes in the hopes of exchanging them for money, he obviously does not feel that sense of attachment when someone buys the product.''
\end{itemize}

\subsection{Identity, Consistency, and Commitments}
\begin{itemize}
    \item ''Another interesting effect of Core Drive 4 [...] is that it also drives us to value our own identity and become more consistent towards our past.''
    \item ''In fact, science has shown that the longer we live, the more attached we become to our existing beliefs, preferences, methodologies, and even our own names.''
    \item ''A [...] study [...] revealed that people are much more likely to choose careers that sound similar to their own names.''
    \item ''Moreover people tend to move to places that are similar to their own names too.''
\end{itemize}

\subsubsection{Consistency}
\begin{itemize}
    \item ''Stemming from the attachment to our own identities is the need to behave consistently with our past actions. Dan Ariely describes this as seslf-herding, where we believe something is good (or bad) on the basis of our previous behaviour.''
    \item Examples: 
    \begin{itemize}
        \item ''Often we buy certain items and brands just because we bought them in the past and we would like to stay consistent with our own choices, even though the product or brand may no longer be the best for our current needs.''
        \item Being more likely to believe that a horse is going to win a race just because we bed on it
    \end{itemize}
    \item ''Our need to be consistent with our past can also cause us to do unreasonable things for others.''
\end{itemize}

\subsubsection{Commitments: The Power of Writing Something Down}
\begin{itemize}
    \item ''Our need of consistency becomes even stronger when we create a commitment, especially when we write it down.''
    \item Example: Study were students had to estimate the length of lines. There were three groups. The first just had to think the estimation in their head. The second group had to write it down but without anyone else knowing their estimation and the third group had to write it down and tell the other participants their guess. Afterwards the researchers gave new misleading information that suggested the students' initial estimates were incorrect and gave them a chance to change their answers. The group that just had to think their estimation were the most likely ones to change their result. Those who had to write it down were not so likely to change their estimation and the group that had to write it down and say it out loud was even unlikelier to change their opinion.
\end{itemize}

\subsection{Game Techniques within Ownership \& Possession}

\subsubsection{Build-From-Scratch (Game Technique \#43)}
    \begin{itemize}
        \item ''When you create a product or service, it is often desirable for your users to increase their vested ownership in the process of its creation. This is why it is useful to have them involved in the development process early on [...].''
        \item ''Building something from scratch means that instead of giving them the entire setup - giving them the the fully furnished house [...], you want them to start off decorating the house from scratch.''
        \item ''When people are building something from scratch, they feel like, ''I own this. This is my thing.''
        \item ''If the Build-From-Scratch technique distracts people away from the first Major Win-State (where users first exclaim, ''Wow! This is awesome!''), then it is not a good design. Either you should give users the option to Build-From-Scratch with some quick template options that will allow them to move forward  quickly and customize later, or you want to ensure that the Build-From-Scratch Technique itself is a First Major Win-State that users feel exited about.''
    \end{itemize}   
    
\subsubsection{Collection Setts (Game Technique \#16)}
    \begin{itemize}
        \item ''One  of the most powerful and effective ways to utilize Core Drive 4''
        \item ''Say you give people a few items, characters or badges, and you tell them that this is part of a collection set that follows a certain theme. This creates a desire in them to collect all the elements and complete the set.''
        \item ''When you give users rewards, don't just give them physical items directly, for those generally have less motivational longevity. More often, giving them collection pieces will result in longer-term engagement.''
        \item ''When a user expects a full reward either due to your own advertising or because of what your competitors are advertising, giving them a Collection Set piece can sometimes backfire and end up insulting that user.''
        \item Example: Geomon, Pokémon, McDonald's Monopoly Game
    \end{itemize}
    
\subsubsection{Exchangeable Points (Game Technique \#75)}
    \begin{itemize}
        \item ''Users can utilize their accumulated points in a strategic and scarce manner to obtain other valuables. Exchangeable Points can have various types of uses. There are no points that can only be redeemed within the game economy for valuables, or points that can be traded with other players in the same system. Some exchangeable points allow users to trade with people outside of the gamified system where they were originally earned in. Each of these types of points has pros and cons, and many good gamified systems [...] have a combination of the above to ensure their economy maintains its value for its users.''
        \item ''It is difficult to run an effective economy. You have to carefully consider the correct labor-to-time-to-tradability-to-reward ratios while constantly adjusting the balance to ensure that people continue valuing your points and currency system. If the system no longer rewards the appropriate labor with the commensurate perceived value, then the economy loses its legacy.''
        \item ''One of the key points to pay close attention to is the scarcity control of the economy. This means that users should never feel like the exchangeable currency or goods are excessively abundant. This is often controlled by the true scarcity of time where labor in the system is balances against the resulting rewards.''
    \end{itemize}
    
\subsubsection{Monitor Attachment (Game Technique \#42)}
    \begin{itemize}
        \item ''Monitor Attachment is a game technique that allows people to develop more ownership towards something, such that they are constantly monitoring or paying attention to it.''
        \item ''When users are monitoring the state of something, they naturally want that state to continually improve. If you are constantly looking at the progression of some numbers, you automatically grow more engaged with the success and growth of these numbers.''
        \item E.g. it can ''be identified in the relationship developed over time with a favorite local coffee shop'' (amongst other things). $\Rightarrow$ ''You build familiarity with the shop, which in turn makes you feel like you partially ''own'' the place as a committed community member and customer.''
        \item 'This is the ''tendency of liking something we feel familiar with''.
        \item ''Because our subconscious minds are bad at differentiating between things that are safe, comfortable, desirable, our brain automatically associates it with something that is safe and desirable. Cognitive ease plays a substantial part towards what we decide to care about and spend time doing.''
        \item ''Paying attention to any possible changes '' is ''incorporating Core Drive 7: Unpredictability \& Curiosity''
        \item ''When one spends a lot of time  monitoring the outcome of something, they will likely develop new ways to improve those outcomes, developing into even more engaged activities within Empowerment of Creativity \& Feedback.''
        \item ''If you could design your experience where users are constantly monitoring the progressive output of something (even if the numbers are going down sometimes), you have a good shot at absorbing the user into much deeper levels of ownership and engagement.''
        \item Example: Tool for monitoring = Google Analytics
    \end{itemize}
    
\subsubsection{The Alfred Effect (Game Technique \#83)}
    \begin{itemize}
        \item ''When users feel that a product or service is so personalized to their own needs that they cannot imagine using another service.''
        \item ''Trough Big Data, we are now able to provide users that sense of personalization by tailoring options based on what smart systems collect on users preferences and habits.''
        \item ''When a user feels like a system has been fully customizing to fit their needs, even if another service out there offered better technologies, functions, or prices, the user has a strong tendency to stay with this system because it now uniquely understands them.''
        \item Examples:
        \begin{itemize}
            \item Amazon: ''Understands your preferences based on all your activities and recommend different products to you''
            \item Google Search: ''Shows personalized search results based on your search and browsing history.''
            \item Waze: navigation app
        \end{itemize}
    \end{itemize}
    
    
\subsection{Core Drive 4: The Bigger Picture}
\begin{itemize}
    \item It ''is a powerful motivator that can attract us to do many irrational things but could also give us great emotional comfort and a sense of well-being. Often it is a central focus that works closely with many of the other Core Drives.''
    \item As one of the more Extrinsically Motivating Core Drives, if improperly designed, it could make people act more selfishly, remove intellectual curiosity, and destroy any hopes of higher creativity.''
\end{itemize}