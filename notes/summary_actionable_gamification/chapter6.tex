\begin{itemize}
    \item ''This is the Core Drive where people are driven by a sense of growth and a need to acomplish a targeted goal. It is what focuses us on a career path, generates our enthusiasm and commitment to learning a new skill, and ultimately motivates us by showing us how far we've come and how much we've grown.''
    \item ''This is also the most common implementation of gamification we see in the market, as most of the PBLs - points, badges, and leaderboards - appera heavily in this drive.
\end{itemize}

\subsection{Development \& Accomplishment in Games}
\begin{itemize}
    \item ''Almost all games show you some type of progress towards the Win-States. A Win-State is often a scenario where the user has overcome some sort of challenge [...]. Games break down user challenges into stages to help the user feel like there is always progress.''
    \item ''Our brains have a natural desire to achieve goals and to experience growth in order to feel that real progress in life is being made.''
    \item The key to Core Drive 2 [...] is to make sure users are proud of overcoming the challenges that are set out for them. Jane McGonigal, renowned game designer and Ph.D. in Performance Studies, defines games as ''unnecessary obstacles that we volunteer to tackle.'''' (book from her ''Reality is broken'' penguin book grop ny, 2011)
    \item ''McGonigal points out that challenges and limitations are what makes a game fun.''
\end{itemize}

\subsection{The First Gamification Site that I was Addicted to}
\begin{itemize}
    \item ebay's design elements:
    \begin{itemize}
        \item Rewarding system for sellers according to the amount of items they sold over ebay and how many positive ratings they get.
        \item There is a game element ''Torture Break'' (''a user must wait an interval of time regardless of their actions, a game technique to be explored under Core Drive 6'') when no one bits on your item for a long time.
        \item suspense is created by the biding system - creates a sense of urge.
        \item Sellers get certificates according to their engagement (sold item and good ratings).
        \item Buyers feel like they have won when retrieving an item because they had to go against other people. They are even paying more than the item is worth sometimes.
        \item Quickly outbiding each other before the time runs out contains 2 game techniques. ''The effect is a combination of a Coundown Timer and a Last Mile Drive, where users feel that they are so close to the goal that they rush to complete it. [...] These mostly employ Black Hat techniques.''
    \end{itemize}
\end{itemize}

\subsection{Never make Users Feel Dumb}
\begin{itemize}
    \item ''Beyond improving one's ranks and obtaining badges, a very important type of emotional accomplishment is to ''feel smart''.''
\end{itemize}

\subsection{Star of Bethlehem - Guiding Users Forward}
\begin{itemize}
    \item ''Feeling a sense of progress and ultimately losing is much better than feeling stuck and confused. If you playthe game through and lose, your natural reflection is to start a new game; but if you get stuck [..] for a long period of time, you may just abandon the game altogether and start doing other things.''
    \item To prevent this from happening you could use the game technique ''Glowing choice, where a user is visually guided by obvious signs towards how to proceed.''
    \item This shown possibility of process does not have to be the best choice and could even lead to failure when chosen (repeatedly). E.g. Candy Crush help.
    \item You can also mark directions or choices you want the user to make.
    \item ''It is important to note that this Desired Action is'' marked in a special way. For this we use the game technique Desert Oasis, ''where visually nothing else is prominent besides the main Desired Action.''
    \item Example for this is an item page of Amazon. Only the Buy option are marked colorful or when looking at a book there often is a big orange arrow to invent people into looking into the book.
\end{itemize}

\subsection{Game Techniques within Development \& Accomplishment}
\subsubsection{Progress Bars (Game Technique \#4)}
\begin{itemize}
    \item ''Our brains hate it when incomplete things are dangled in front of our faces.''
\end{itemize}

\subsubsection{The Rockstar Effect (Game Technique \#92)}
\begin{itemize}
    \item ''The Rockstar Effect is a gamification design technique where you make users feel everyone is dying to interact with them. In essence, if you make people feel like they have raned their way in becoming a Rockstar, they will feel so much pride in that they will continue to perform the Desired Actions of building up an even greater fanbase and sharing with others.''
    \item Example: Twitter's one-way follow.\\
    ''Many people saw getting many followers as a true achievement - meaning that everyone wanted to listen to your valuable opinions'' while you could ''ignore'' them.
    \item ''At one point, influential people started to compete with eachh other to see who had more followers.''
\end{itemize}

\subsubsection{Achievement Symbols (Game Technique \#2)}
\begin{itemize}
    \item ''Bages are what I call ''Achievement symbols'' and can come in many forms - badges, stars, belts, hats, uniforms, trophies, medals, etc.. The important thing about Achievement Symbols is that they must symbolize ''achievement''.''
    \item ''If trough your creative skills you solved a unique problem that not everyone could solve, and as a result received a badge to symbolize that achievement, you feel proud and accomplished.''
    \item ''Achievement symbols merely reflect achievement, but are not achievement by themselves.''
\end{itemize}

\subsubsection{Status Points (Game Technique \#1)}
\begin{itemize}
    \item ''There are two types of points in a motivation system: Status Points and Exchangeable Points. Status Points are for keeping score of progress. Internally, it allows the system to know how close players are towards the win-state. Externally, it gives players a feedback system for tracking their progress.''
    \item ''Showing people their score and how it changes based on small improvements often motivates them towards the right directions.''
    \item ''Within Status Points, there are also smaller divisions of types. For example, Absolute Status Points (which measure the total amount of points earned during a journey) versus Marginal Status Points (which are points that are specifically set for a given challenge or one time period, and can be reset once that challenge or time period is over).''
    \item ''One-Way Status Points (points that can only go up) versus Two-Way Status Points (it can also go down as the user fails to achieve the Win-State).''
    \item ''How you craft the gain and loss of points, as well as meaning behind the points can significantly change the users' perception of your product.''
    \item ''With the rules you set, you are establishing an interaction with the user and communicating your values.''
    \item ''If you give people a bunch of points just to do marketing for you, or reward them with virtual items for every little stupid thing, users will feel like the game is shallow.''
    \item ''Users have no interest in a game if they know the game designer is just trying to benefit themselves instead of caring about their community.''
    \item ''When you design your Status Point system, make sure it is based om something meaningful - something that the users themselves want to stress people out.''
\end{itemize}

\subsubsection{Leaderboards (Game Technique \#3)}
\begin{itemize}
    \item ''Leaderboards is a game element where you rank users based on a set of criteria that is influenced by the users' behaviour towards the Desired Action.''
    \item ''If you use a site for a few hours and receive 25 points, and then see the Top 20 list that number 20 already has 25,000,000 points, that would likely discourage you from trying further.''
    \item ''What users need is Urgent Optimism, another term coined by Jane McGonigal, where the user feels optimistic that they can accomplish the task, but also the urgency to act immediately.''
    \item ''There are a couple of variations that have shown to perform more effectively:
    \begin{itemize}
        \item ''First, you always want to position the user in the middle of the leaderboard display, so all they see is the player ranked right above them and the player ranked just below. It's not very motivating in seeing how high the Top 10 players are, but it's incredibly motivating when one sees someone who used to be below them suddenly excelling.''
        \item Another variation that has proved successful is to set up Group Leaderboards where the ranking is based on the combined efforts of a team. In this case, even though not everyone is competitive and needs to be at the top, most people don't want to be laggard that drags the team down. As a result everyone works harder because of Social Influence \&Relatedness (Core Drive 5).''
        \item ''The next variation is to set up constantly refreshing leaderboards, where every week the data would refresh and the leaderboards will start tracking progess anew.''
        \item Finally, it's a good idea to implement micro-leaderboards, where only the users' friends or very similar people are compared.''
    \end{itemize}
    \item ''The key way to effectively integrate a leaderboard is to ensure that the user can quickly recognize the action items that drives them to reach win-state. If there is no way of achievement, there is no action.''
\end{itemize}

\subsection{Core Drive 2: The Bigger Picture}
\begin{itemize}
    \item ''Since Development \& Accomplishment is the easiest Core Drive to design for, many companies focus on this Core Drive sometimes almost exclusively.''
    \item '' Is often a natural result after good implementation of other Core Drives, such as Core Drive 3: Empowerment of Creativity \& Feedback, Core Drive 4: Ownership \& Possession, as well as Core Drive 6: Scarcity \& Impatience.''
    \item ''Often, it also leads to Core Drive 5: Social Influence \& Relatedness, where the user wants to share with their friends that sense of achievement and accomplishment.''
\end{itemize}