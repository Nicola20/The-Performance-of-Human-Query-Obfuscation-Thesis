\subsection{Explicit Gamification: Games that Fulfill Non-Game Purposes}
\begin{itemize}
    \item ''The two types of gamification implementations in my own work are ''Implicit Gamification'' and ''Explicit Gamification''.
    \item ''Explicit Gamification involves strategies that utilize applications that are obviously game-like. Users acknowledge they are playing a game, and generally need to pot into playing.''
    \item Examples for Explicit gamification: 
    \begin{itemize}
        \item Dikembe Mutombo's 4 1/2 weeks to Save the World
        \item McDonald's Monopoly Game
        \item FoldIt
        \item Undiscovered Territory
        \item Repair the Rockaways
    \end{itemize}
    \item ''The advantage of designing for explicit gamification is that the product is generally more playful and it allows the designer to have more freedom of creativity. The disadvantage is that it could be seen as childish, non-serious, or distracting to some target users such as enterprise firms, banks or manufacturers.''
\end{itemize}

\subsection{Implicit Gamification: Human-Focused Design that Utilizes Game Elements}
\begin{itemize}
    \item ''Implicit Gamification is a form of design that subtly employs gamification techniques and the 8 Core Drives of Octalysis into the user experience. Implicit Gamification techniques are filled with game design elements that are sometimes even invisible to the user.''
    \item Examples:
    \begin{itemize}
        \item LinkedIn Progress Bar
        \item intrinsic motivation that drives Wikipedia
        \item competitive bidding system and feedback system via eBay
        \item social comparison and motivation in OPower
        \item Unpredictability and scarcity within Woot!
    \end{itemize}
    \item '' The advantage of implicit gamification is that it is technically easier to implement and can be approriate in most contexts. The disadvantage [...] is that this very convenient implementation can often lead to ''lazy'' design where the subtle game dynamics are incorrectly designed for and sloppily put together. This would lead to something completely ill-formed or ineffective in terms of driving business metrics.''
\end{itemize}

\subsection{Implicit vs. Explicit Gamification}
\begin{itemize}
    \item ''One type of gamification is not inherently better than the other.''
    \item ''The proper use of Implicit or Explicit Gamification depends on the purpose of the project as well as your target market.''
    \item ''All 8 Core Drives can be implemented into Implicit and Explicit Gamification campaigns.''
\end{itemize}